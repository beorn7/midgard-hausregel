\documentclass[10pt,a4paper,germanpar]{article}
\usepackage{german,anysize,eurosym,tabu,enumitem,gitinfo2}
%\usepackage[T1]{fontenc}
\usepackage{helvet}
%\usepackage{mathphv}
\usepackage[utf8]{inputenc}

\renewcommand{\familydefault}{phv}

\renewcommand{\arraystretch}{1.5}

\setlist[itemize]{topsep=0.1pt,itemsep=-0.8ex,partopsep=0.1ex,parsep=1ex}
\setlist[description]{topsep=0.1pt,itemsep=-0.8ex,partopsep=0.1ex,parsep=1ex}
\setlist[enumerate]{topsep=0.1pt,itemsep=-0.8ex,partopsep=0.1ex,parsep=1ex}

\marginsize{1.8cm}{1.8cm}{1.5cm}{1.5cm}
\setlength{\footnotesep}{5mm}

\hyphenation{NSpF}
\hyphenation{Re-gel-mo-di-fi-ka-tio-nen}


\author{Bj"orn~Rabenstein\\ bjoern@rabenste.in}
\title{Hausregeln f"ur
  \textsc{Midgard} (4. Auflage)\\Version\gitReln}
\date{\gitAuthorDate}


\begin{document}

\maketitle

%\tableofcontents

\section{Vorwort}

Regeln sind Nebensache im Rollenspiel. In welcher Hinsicht haben
Regeln "uberhaupt eine Berechtigung? Ein Rollenspiel ist nicht
realistisch. Manchmal erweckt es den Anschein, realistisch zu sein,
aber wenn man nur hinreichend genau hinschaut, ist man immer beliebig
weit von der Realit"at (was auch immer das sein mag) entfernt. Ein
Rollenspiel hat seine eigene Realit"at. Die Regeln dienen dazu, dem
daran interessierten Spieler ein Gef"uhl zu geben, welchen Gesetzen
diese spezielle Realit"at folgt. Sie dient auch dem SpL,
einsch"atzen zu k"onnen, was die Spieler f"ur eine Erwartung von
dieser Realit"at haben, um also zu vermeiden, da"s bei den Spielern
der Eindruck von Willk"ur oder Ungerechtigkeit entsteht. Schlußendlich
sorgen die Regeln dafür, daß die Spielwelt sich (in gewissen Grenzen)
"`selber spielt"' -- das Rollenspiel erhält in Teilen den Charakter
einer Simulation, vergleichbar einem simulationsbetonten Brett- oder
Computerspiel. Von all dem unbeschadet bleibt die "`goldene"' Regel,
da"s der SpL immer recht hat und erhaben "uber jede Regel ist
(was selbstverständlich nur behutsam Anwendung finden sollte, da
andernfalls der Zweck der Regeln \emph{ad absurdum} geführt würde). Die
hier folgenden Klarstellungen und "Anderungen sollen in diesem Sinne
lediglich vermeidbare "Uberraschungen verhindern, insbesondere bei
Spielern, die sich sehr auf den gedruckten Regeltext verlassen.

Generell folgen diese Hausregeln der Philosophie, da"s
"`echter"' Realismus im obigen Sinne ohnehin nicht m"oglich ist und es
daher zwecklos ist, mit hochkomplexen Regeln diesem Realismus
nachzueifern, ohne ihm wirklich n"aher zu kommen, dabei allerdings die
Spielbarkeit des Systems immer mehr zu untergraben. Insbesondere die
Kampfregeln sind ein gutes Beispiel daf"ur. Die Regeln sollen hier
eine anschauliche Methode zur Verf"ugung stellen, K"ampfe spielerisch
umzusetzen und den SpF taktische Entscheidungen und die Steuerung
ihrer Kampfhandlungen zu erm"oglichen. Von einer wirklich
realistischen, detailierten Simulation kann nicht die Rede sein. Daher
wurde hier gar nicht erst versucht, die Regeln durch erh"ohte
Komplexit"at "`realistischer"' zu gestalten, sondern nur an den
wenigen Stellen behutsam einzugreifen, wo es nach subjektivem Eindruck
an Spielbarkeit, Klarheit, Konsistenz oder Nachvollziehbarkeit
mangelte. Echte Regelerg"anzungen mu"sten sich mit einem deutlichen
Gewinn an Spielbarkeit und Spielqualit"at, Konsistenz und
Nachvollziehbarkeit rechtfertigen, nicht jedoch mit einem Gewinn an
"`Realismus"'.

Diese Klarstellungen, Änderungen und Ergänzungen der offiziellen
\textsc{Midgard}-Regeln sind den Erfahrungen un\-zäh\-liger Sitzungen
seit 1987 entwachsen. Aufgeschrieben wurden sie von mir, aber ich hege die
Hoffnung, daß sie dem Gruppenkonsens Ausdruck verleihen.

Seit Version~2.0 beziehen sich diese Hausregeln auf die 4.~Auflage des
\textsc{Midgard}-Regelwerks. Verweise auf das Regelwerk werden in der Form
[XXX\,nnn] gegeben, wobei XXX ein Kürzel für das jeweilige Buch ist,
während nnn die Seitenzahl angibt. Folgende Kürzel finden Verwendung:
DFR = \emph{Das Fantasy-Rollenspiel}, ARK = \emph{Das Arkanum}, KOM =
\emph{Das Kompendium}, BES = \emph{Das Bestiarium}, MDS =
\emph{Meister der Sphären}, KTP = \emph{Unter dem Schirm des
  Jadekaisers} (KanThaiPan-Quellenbuch). Der Begriff \emph{Figur} (als
"`altdeutsche"' Version von englisch \emph{Character} oder
eingedeutscht \emph{Charakter}) ist in den jüngeren
\textsc{Midgard}-Auflagen irgendwie der Vergessenheit anheim gefallen,
erfreut sich hier aber noch reger Benutzung. Also: SpF = Spielerfigur
= \emph{player character} = PC = "`Abenteurer"' = "`Held"'. NSpF =
Nichtspielerfigur = \emph{non-player character} = NPC. Desweiteren
werden die üblichen \textsc{Midgard}-Abkürzungen benutzt
(vgl. [MDS\,296]). Die Schreibweise \textsc{Midgard} bezieht sich auf
das Regelsystem, während die Spielwelt \emph{Midgard} als Midgard
bezeichnet wird.


\section{Druckfehler}

Auf offensichtliche Druckfehler, die leicht zu korrigieren sind, wird
hier nicht weiter eingegangen. Hier werden nur F"alle erw"ahnt, bei
denen zwar eindeutig ein Druckfehler vorliegt, wo aber nicht ohne
weiteres klar ist, wie dieser Druckfehler zu verbessern ist.

Die bereits veröffentlichten offiziellen Errata werden als Bestandteil
der Regeln betrachtet und hier bereits vorausgesetzt.

\begin{itemize}
\item\ [KOM\,56] In allen vier Tabellen muß jeweils die Zeile
  "`+7,\,+8,\,+9~\textbf{100}"' korrigiert werden zu
  "`+8,\,+9~\textbf{100}"'. (Die +7 steht schon in der jeweils
  vorangehenden Zeile.) 
\item In [DFR\,120] linke Spalte, vorletzter Absatz steht geschrieben, daß ein
  Kurzsichtiger keine Probleme beim Fernkampf hat (mit Ausnahme von
  Scharfschießen).  In [DFR\,116] linke Spalte, vorletzter Absatz steht
  hingegen, daß der SpL vor jedem Schuß auf Ziele außerhalb des
  Nahbereichs einen EW:Sehen verlangen kann. Es gilt letztere Regel, wobei
  ausdrücklich auf das \emph{kann} verwiesen wird.  Der EW ist also kein
  Automatismus, sondern dem SpL überlassen.
\end{itemize}

\section{Geburt eines Abenteurers}

Ergibt der Wurf von 1W6 für LP [DFR\,35] ein Ergebnis von weniger als
4, so wird der Wurf dennoch als 4 interpretiert.
% Backport von M5.

\section{Fertigkeiten}

\subsection{Angeborene Fertigkeiten}

Wird eine \textbf{angeborene Fertigkeit} [DFR\,38f] erwürfelt, die die
SpF aufgrund ihrer Basiseigenschaften nicht haben kann, wird der Wurf
wiederholt. Das gleiche gilt für Fertigkeiten, über die die SpF
bereits verfügt oder die im Widerspruch zu solchen Fertigkeiten stehen
(z.\,B. steht \emph{Riechen+10} mit \emph{Riechen+4} im Widerspruch
(u.\,u.) oder \emph{Kurzsichtigkeit} mit \emph{Nachtsicht}). Im
unwahrscheinlichen Fall, daß keine weitere angeborene Fertigkeit mehr
möglich ist, entfällt der Wurf.

\subsection{Automatisch Informationen erhalten}

Der \textbf{automatische} EW f"ur Entdeckungs- und Wissensfertigkeiten
und für Sinne [DFR\,114] mu"s nicht grunds"atzlich \textbf{kritisch
  erfolgreich} sein. Gelingt der Wurf mit zusätzlicher \textbf{WM--8},
ist dies gleichwertig zu einem kritischen Erfolg zu sehen (in
Anlehnung an M5). Bei einfach zu entdeckenden Informationen kann nach
SpL-Ent\-schei\-dung sogar ein \textbf{normaler Erfolg} reichen, und
ein automatischer EW kann auch gewährt werden, wenn die entsprechende
Fertigkeit ungelernt bzw. der Sinn nicht überdurchschnittlich gut
entwickelt ist. Ein derartiges Vorgehen wird auch in den Regeln an
vielen Stellen mehr oder weniger deutlich angedeutet: [DFR\,78]
(EW:Riechen im Beispiel), [DFR\,114] (Beispielkasten), [DFR\,145]
(Giftmischen), [DFR\,159f] (Menschenkenntnis)\dots

\subsection{Leit- und Mindesteigenschaften}

Eine Fertigkeit kann auch erlernt werden, wenn die SpF die nötigen
\textbf{Mindestwerte von Eigenschaften} nicht erfüllt (entgegen
[DFR\,121]). Allerdings erhält eine SpF \textbf{WM--2}, wenn sie zum
Zeitpunkt des EW nicht die für die Fertigkeit nötigen
Eigenschaftsmindestwerte erfüllt.

Werte ab 81 bei der \textbf{Leiteigenschaft} haben nicht mehr die in
[DFR\,121] angegebenen Vorteile. Stattdessen verleiht die
Leiteigenschaft einen Bonus oder Malus, analog zum AnB durch Gs oder
AbB durch Gw. Die WM--2 aus dem vorigen Absatz gilt allerdings
absolut. Ist sie relevant, wird kein Bonus oder Malus aufgrund der
Leiteigenschaft berücksichtigt.

Für ungelernte Fertigkeiten gilt weiterhin die Regelung in
[DFR\,57]. Es gibt also WM--2 oder gar keine Modifikation.

Für \textbf{Sprechen} gibt es niemals einen Malus. Für
\textbf{Schreiben} gibt es nur bei weniger als \textbf{In21} einen
Malus.

Für \textbf{Trinken} verleiht Ko keinen Bonus (bzw. kann man
\textbf{+Ko/10} als den Bonus ansehen).

Der Fertigkeitsbonus auf \textbf{Werfen} entspricht dem AnB.

Beim \textbf{Beidhändigen Kampf} führt der Fertigkeitsbonus effektiv
dazu, daß nun indirekt doch eine Art AnB berücksichtigt wird
[DFR\,133], allerdings auf Gw basierend statt Gs. Dies ist aus
Konsistenzgründen erwünscht. (Zum Ausgleich fallen ja auch die in
[DFR\,121] angegebenen Vorteile weg.)

Genauso wie AnB und AbB durch schwere Rüstung reduziert werden [DFR\,98]
kann der SpL in passenden Situationen den durch Gs bzw. Gw verliehene
Fertigkeitsboni in gleichem Maße reduzieren. Dies ist dann sinnvoll,
wenn der Einsatz der Fertigkeit ähnlich durch Rüstung beeinträchtigt
wird wie Angriff bzw. Abwehr \emph{und} Rüstung nicht bereits durch
andere WM berücksichtigt wird.

\subsection{Einzelne Fertigkeiten}

Die Fertigkeiten \textbf{Alchimie} [DFR\,130], \textbf{Giftmischen}
[DFR\,145f], \textbf{Heilkunde} [DFR\,147f], \textbf{Kr"auterkunde}
[DFR\,156f], \textbf{Pflanzenkunde} [DFR\,164], \textbf{Tierkunde}
[DFR\,188] haben eine mehr oder weniger große Schnittmenge im Gebiet
der Verarbeitung und Identifikation von Substanzen. Die folgenden
Hinweise helfen bei der Abgrenzung der Fertigkeiten:
\begin{itemize}
\item Generell sollte sich der SpL eher großzügig zeigen, was die
  Anwendungsgebiete der Fertigkeiten angeht, insbesondere wenn die Spieler
  fanatasievoll darlegen können, wie ihre Figuren eine bestimmten Fertigkeit
  in einer gegebenen Situation geeignet anwenden können. Andererseits sollte
  eine SpF, die gleich mehrere potentiell anwendbare Fertigkeiten beherrscht
  (z.\,B. \emph{Heilkunde, Giftmischen, Kräuterkunde, Pflanzenkunde} oder gar
  \emph{Tierkunde} bei der Frage, ob und wie ein bestimmtes Kraut gegen
  einen Wespenstich angewendet werden kann), nicht benachteiligt werden.
  Einerseits kann man bei Anwendung einer nicht genau passenden Fertigkeit
  negative WM verteilen, andererseits kann man einer SpF, die mehrere
  passende Fertigkeiten beherrscht, mehrere EW zugestehen (gewissermaßen für
  verschiedene Herangehensweisen an dasselbe Problem). (Siehe hierzu
  auch [DFR\,110].)
\item \emph{Alchimie} ist die allgemeinste der angesprochenen Fertigkeiten.
  Sie ist gewissermaßen genau das Fachgebiet, in dem all die anderen
  angesprochenen Fertigkeiten ihre Schnittmenge haben (Verarbeitung und
  Identifikation von Substanzen). Grundsätzlich sollte Alchimie also dann
  Anwendung finden, wenn keine der anderen Fertigkeiten so recht passen
  will. In den anderen Fällen kann Alchimie ebenfalls helfen, aber dann ist
  mit negativer WM zu rechnen. Je näher eine (ehemals) belebte Substanz noch
  ihren natürlichen Ursprüngen ist, desto eher finden \emph{Pflanzenkunde},
  \emph{Tierkunde} oder auch \emph{Kräuterkunde} Anwendung. Wenn das
  Anwendungsgebiet in Richtung Heilung bzw. Vergiftung tendiert, dann passen
  eher \emph{Kräuerkunde} und \emph{Heilkunde} bzw. \emph{Giftmischen}. Die
  Verwendung unbelebter oder bereits stark alchimistisch bearbeiteter
  Substanzen, die gleichzeitige Verwendung von Substanzen
  unterschiedlichsten Ursprungs und die Anwendung von Magie sind Indizien
  für \emph{Alchimie}. Auch die Methodik kann bei der Entscheidung eine
  Rolle spielen. Dazu halte man sich vor Augen, daß die Alchimie in der
  realen Welt die Vorläuferin der Chemie darstellt, aber auch mit viel
  Mystik und Esoterik durchdrungen ist. (Etwas davon erahnen konnte man in
  \emph{Sturm über Mokattam}, wo die "`chymische Hochzeit"' der
  gegensätzlichen Prinzipien Schwefel und Quecksilber zum roten Zinnober
  vorkam.)
\item \emph{Giftmischen} ist gewissermaßen ein Spezialgebiet, das aus allen
  anderen angesprochenen Fertigkeiten "`entnommen"' wurde. Die Verarbeitung
  und Identifizierung tierischer Substanzen zu Giften stammt aus der
  \emph{Tierkunde}, entsprechendes gilt für pflanzliche Substanzen und
  \emph{Kräuterkunde} und \emph{Pflanzenkunde}. Die Herstellung und
  Anwendung von Gegengiften überschneidet sich mit \emph{Heilkunde}
  und \emph{Kräuterkunde}, während die Verarbeitung und Identifizierung von
  unbelebten oder alchimistisch verarbeiteten giftigen Substanzen der
  \emph{Alchimie} entstammt. Für magische Gifte gilt dies umso mehr.
\item Die \emph{Heilkunde} beschäftigt sich natürlich hauptsächlich nicht
  mit der Anwendung und Identifizierung von Substanzen, sondern mit ganz
  anderen Dingen (Identifikation von Krankheiten, "`handwerkliches"'
  Vorgehen zur Wundbehandlung und vieles mehr). Dennoch wird ein
  Heilkundiger sich nicht Kenntnissen verschließen, welches Kraut bei einem
  Fieber Anwendung finden könnte oder welches Gegengift nach dem Biß einer
  Giftschlange wirken könnte (oder auch nur, welche Symptome auf ein
  bestimmtes Gift hinweisen). Hier gibt es also durchaus eine Schnittmenge.
  Allerdings ist die Heilkunde diesbezüglich recht unspezifisch, wie dies ja
  auch im Regelwerk schon angedeutet wird.
\item Die \emph{Kräuterkunde} wird auch im Regelwerk schon relativ klar von
  \emph{Giftmischen} und \emph{Heilkunde} abgegrenzt. Problematischer ist
  die Abgrenzung zur \emph{Pflanzenkunde}. Generell ist die
  \emph{Pflanzenkunde} eher für die Identifikation und das Auffinden von
  Pflanzen in der freien Natur geeignet, während die \emph{Kräuterkunde}
  eher die Weiterverarbeitung und Anwendung der pflanzlichen
  Substanzen beinhaltet, und zwar insbesondere wenn es um eine heilende
  Wirkung geht. Hat eine Pflanze schon in Rohform und ohne größere Umstände
  bei der Anwendung eine heilende Wirkung, wird dies auch ein
  Pflanzenkundiger wissen (Beispiel: schmerzstillende Wirkung durch Kauen
  von Rinde der Trauerweide).
\item Die \emph{Tierkunde} beschäftigt sich ebenso wie die \emph{Heilkunde}
  nur nebenbei mit den hier besprochenen Aspekten der Identifizierung und
  Verarbeitung von Substanzen. Aber ebenso wie \emph{Kräuterkunde} und
  \emph{Pflanzenkunde} im Bereich pflanzlicher Substanzen kann
  \emph{Tierkunde} im Bereich tierischer Substanzen angewendet werden. Der
  Übergang zu \emph{Alchimie} (bei weitergehender Verarbeitung tierischer
  Substanzen, insbesondere magischer Natur) bzw. zum \emph{Giftmischen} (bei
  Verwendung tierischer Substanzen zum Mischen von Giften und Gegengiften)
  ist eher früher anzusetzen als bei \emph{Kräuter-/Pflanzenkunde}, da
  Tierkunde sich wie gesagt nur am Rande mit der Verarbeitung und
  Identifizierung tierischer Substanzen beschäftigt.
\end{itemize}

Erfolgreich eingesetzte \textbf{Akrobatik} ermöglicht einen spontanen
Angriff in derselben Runde, in der das Aufstehen erfolgte (jedoch
nicht ein Aufstehen in derselben Runde, in der der Sturz erfolgte --
die Formulierung in [DFR\,129] ist mißverständlich).

Beim mit \textbf{Beidhändigem Kampf} durchgeführten
\emph{Kombinationsangriff} zählt der jeweils für den Angreifer
ungünstigere Waffenrang und Rüstungsbonus der beiden eingesetzten
Waffen.

Der EW zum Überprüfen, ob das \textbf{Einprägen} [DFR\,117] gelungen
ist, wird erst beim Versuch des Erinnerns gewürfelt. (Dies soll
verhindern, daß eine SpF einfach solange hintereinander
\emph{Einprägen} anwendet, bis der EW gelingt.)

Abweichend vom Regelwerk spart eine SpF mit
\textbf{Geschäftstüchtigkeit} bei nicht ausgespielten Lernphasen
grund\-sätz\-lich 10\,\%\ der Lernkosten, ohne daß dafür ein EW fällig
wäre. (Im Gegensatz zum Regelwerk betrachten die Hausregeln die
Lernkosten eben nicht nur als direkte Bezahlung des Lehrmeisters,
sondern auch als Kosten für den Lebensunterhalt.)

Nach [DFR\,160ff] mu"s beim \textbf{Meucheln} mindestens 1\,LP
schwerer Schaden angerichtet werden, damit das Opfer stirbt. Je nach
Methode des Meuchelns und Art der R"ustung wird der R"ustungsschutz
jedoch nicht ber"ucksichtigt. (Einen Dolch durch eine Plattenr"ustung
ins Herz zu sto"sen ist fast unm"oglich. Beim Erw"urgen mit einer
Schnur nutzt die Plattenr"ustung hingegen "uberhaupt nichts. Eine
Textilr"ustung oder Lederr"ustung bietet vielleicht eine passende
L"ucke f"ur einen Sto"s ins Herz.) Die negative WM--Grad, die auf den
EW:Meucheln angerechnet wird, steht f"ur die besseren Reflexe, die
Intuition und den sechsten Sinn, die erfahrene Figuren sich angeeignet
haben. Bei Kreaturen, die aufgrund ihrer angeborenen Gef"ahrlichkeit
und nicht aufgrund von Erfahrung einen hohen Grad besitzen, wird diese
WM abgeschw"acht oder gar aufgehoben.

Das Regelwerk sieht offenbar vor, daß beim \textbf{Beschleichen} eines
\textbf{Schlafenden} kein \textbf{EW:Schleichen} erforderlich
ist. Damit ist die in [DFR\,95] beschriebene Situation, in der ein
\emph{bewegungsloser} Gegner automatisch kritisch (bzw. ohne Zeitdruck
und bei ungestörtem Angreifer sogar tödlich) getroffen wird, recht
leicht herbeiführbar (z.\,B. mit Hilfe des Zaubers
\emph{Schlaf}). Nicht zuletzt zum Schutz der SpF wird daher hier
geregelt, daß zum Beschleichen von Schlafenden ein EW:Schleichen
erforderlich ist. Der Schleichende erhält dabei allerdings
\textbf{WM+4} (wie \emph{Ablenkung durch schwierige
  T"atigkeit}). Außerdem gilt der Schlafende als besonders
unaufmerksam (erwacht nur nach gelungenem WW:\emph{Hören}). Um
\emph{Wachgabe} [DFR\,121] durch diese Regelung nicht zu entwerten,
gilt folgendes: Wenn einem Schlafenden mit \emph{Wachgabe} der EW
gelingt, wird er beschlichen, als wenn er wach wäre, d.\,h. es gibt
keine WM+4 für den Schleichenden, und er wacht bei Mißlingen des
EW:\emph{Schleichen} auf jeden Fall auf.

Ein \textbf{Taucher} bewegt sich mit \textbf{B3}. Wenn er sich
schneller bewegt, werden die \textbf{EW:Tauchen} entsprechend fr"uher
f"allig. [DFR\,188]

\section{Erfahrung}

\subsection{Vergabe von Erfahrungspunkten}

\subsubsection{AEP}
\label{aep}

Auch bei der Vergabe von AEP für die kampflose Überwindung eines
Gegners [DFR\,270f] findet der für die KEP-Vergabe gültige
Multiplikator Anwendung. (Faustregel: Die Anzahl an AEP für die
Überwindung eines Gegners entspricht der Anzahl an KEP, die es gegeben
hätte, wenn dem Gegner alle AP durch Fernkampfangriffe geraubt worden
wären.)

\subsubsection{KEP}

In der Regel verliert ein Gegner AP entsprechend des erw"urfelten
Schadens. Es gibt jedoch auch indirekte AP-Verluste (z.\,B. wenn durch
den Schlag der Gegner auf die H"alfte seiner LP reduziert wird und
dadurch auch nur noch maximal "uber die H"alfte seiner AP verf"ugt)
und die M"oglichkeit, da"s der Gegner nach dem Schlag kampfunf"ahig
ist. F"ur diese Auswirkungen gibt es nur unter bestimmten Umst"anden
KEP. Bei einem \textbf{normalen Treffer} gibt es immer nur maximal
f"ur den \textbf{direkt erw"urfelten Schaden KEP} (da"s ein Gegner
durch die LP-Verluste dieses Treffers auf die H"alfte seiner AP
reduziert wird oder schwer verletzt zusammenbricht, ist gewisserma"sen
Zufall und erh"oht nicht die durch diesen Treffer gemachte
Erfahrung). Wird ein Gegner durch einen \textbf{kritischen Treffer},
einen \textbf{gezielten Hieb} oder \emph{Faustkampf} get"otet,
bewu"stlos oder v"ollig kampfunf"ahig, so erh"alt die SpF ("ahnlich
wie ein Zauberer, der einen Gegner durch einen einzigen Zauber
kampfunf"ahig macht) \textbf{soviele KEP}, als wenn er dem Gegner
\textbf{alle seine verbliebenen AP} durch normale Angriffe geraubt
h"atte.  \textbf{Keine} KEP gibt es f"ur Angriffe, bei denen der
Gegner zu keiner Gegenwehr f"ahig ist, insbesondere
\textbf{Fernkampfangriffe aus dem Hinterhalt} (also auch beim
Scharfschie"sen) oder Angriffe beim \textbf{Meucheln}
[DFR\,268]. Hierf"ur gibt es unter Umst"anden \textbf{AEP} f"ur die
"Uberwindung eines Gegners (siehe Abschnitt \ref{aep}).

\subsubsection{ZEP} 

Gelingt sowohl der EW:Zaubern wie auch der WW:Resistenz, erh"alt der Zauberer
in der Regel dennoch ZEP, sofern der Zauber nicht ausschließlich darauf
gezielt hat, dem Opfer AP zu rauben bzw. es kampfunfähig zu machen [DFR\,269].
Allerdings sollte der SpL bei Serien von vergeblichen Bannversuchen
o.\,ä. nur einen Versuch mit ZEP belohnen. (Wenn aber dann das Zauberduell
gegen einen überlegenen Zauberer schließlich doch gelingt, darf es ruhig ein
paar Bonus-ZEP geben.)

Ob ein Zauber ein Angriffszauber oder ein sonstiger Zauber ist, wird
flexibel entschieden. Z.\,B. kann eine Feuerkugel sehr wirksam die
Gegner erschrecken oder gar vertreiben, ohne einen einzigen AP Schaden
anzurichten. In einem solchen Fall z"ahlt \emph{Feuerkugel} als
sonstiger Zauber.  Beschw"orungen werden grunds"atzlich als sonstige
Zauber gez"ahlt.  F"ur effiziente Anwendung der beschworenen Wesen
kann der Beschw"orer jedoch AEP erhalten. ZEP f"ur \textbf{sonstige
  Zauber}, bei denen der Zauberer \textbf{alle} seine AP opfern mu"s,
werden grunds"atzlich nicht nach der Anzahl der verbrauchten AP
berechnet, sondern vom SpL je nach Bedeutung des
durchgef"uhrten Zaubers frei bestimmt.

\subsection{Praxispunkte}

Bei PP, die nach den Standardregeln durch einen \textbf{doppelten EW}
vergeben werden, wird grundsätzlich ein PP vergeben, wenn das
unmodifizierte Würfelergebnis bei einem erfolgreichen EW 16--20
beträgt. Es wird also nicht mehr "`hinterhergewürfelt"' für den
PP. Außerdem kann der SpL nach einem „besonderen“ EW (der besonders
entscheidend, dramatisch, erfahrungsträchtig, rollengerecht, \dots
war), einen „Bonus-PP“ vergeben. Je nach Situation kann er dabei sogar
nach einen mißlungenen EW einen PP vergeben ("`auch aus Fehlern kann
man lernen"' oder auch im Sinne der Variante für Wissensfertigkeiten
[DFR\,278]).
% Z.T. Backport von M5.

% Die Wahrscheinlichkeiten, die sich durch die Anwendung
% der Standardregeln für die PP-Vergabe ergeben hätten, können dabei
% durchaus eine Richtlinie sein (einem Experten bringt eine einfach
% Routineaufgabe kaum noch Praxiserfahrung), aber die letztendliche
% Kontrolle bleibt beim SpL.
% 
% \emph{Anmerkung: Diese Regeländerung dient zwei Zwecken: Zum einen
%   wird die Anzahl an nötigen Würfelwürfen reduziert, was den
%   Spielablauf weniger stört. Zum anderen sind die PP unserer Erfahrung
%   nach besonders für "`Zwischencharaktere"' wichtig, also solche, die
%   viele allgemeine Fertigkeiten beherrschen, aber weder besonders gut
%   kämpfen noch zaubern können. ZEP und KEP fließen auf hohen Graden
%   reichlich, aber es bleibt bei 5~AEP für die Anwendung allgemeiner
%   Fertigkeiten. Die PP können hier für einen gewissen Ausgleich
%   sorgen. Mit der 4.~Auflage der \textsc{Midgard}-Regeln wurde das
%   PP-System neu geregelt. Dabei wurde allerdings das Gleichgewicht
%   zu Ungunsten der allgemeinen Fertigkeiten verschoben. Mit dieser
%   Hausregel hat es der SpL in der Hand, für Ausgleich zu sorgen.}

\subsection{Lernen}

Die Lernregeln geh"oren wohl zu den meistdiskutierten Teilen des
\textsc{Midgard}-Systems. Immer wieder werden die astronomisch hohen
Lernkosten kritisiert oder die fehlende M"oglichkeit, da"s sich SpF
gegenseitig etwas beibringen. Dieser Abschnitt f"uhrt einige wenige
Neuregelungen ein, dient aber viel eher der Interpretation als der
"Anderung der existierenden Regeln.

\subsubsection{Grundsätzliche Anmerkungen}

Die Lernregeln sind eine pauschale Regelung, um die sonst in vielen
F"allen schnell stereotypisch ablaufenden Lernvorg"ange nicht explizit
ausspielen zu m"ussen. So eine Pauschalregelung kann in Einzelf"allen
zu unsinnigen Ergebnissen f"uhren. In solchen F"allen ist flexibel zu
improvisieren. Beim Erlernen besonderer F"ahigkeiten, also z.\,B.
eines speziellen Zauberspruchs oder eines besonders hohen
Erfolgswertes, bietet es sich an, das Aufsuchen des Lehrmeisters
auszuspielen. F"ur m"achtige Zauberspr"uche und hochperfektionierte
Fertigkeiten gibt es nur wenige Lehrmeister, die in Frage kommen.
Diese einzigartigen Lehrmeister sind auch nicht unbedingt mit Geld zu
bezahlen, sondern eher in ebenso einzigartigen Dienstleistungen oder
einzigartigen wertvollen, wom"oglich magischen Gegenst"anden. Die
hohen Lernkosten erkl"aren sich auch dadurch, da"s Geld eigentlich das
falsche Mittel ist, herausragende Lehrmeister zu bezahlen. Aus der Not
kann eine Tugend gemacht werden, denn die Bezahlung in
Dienstleistungen ist ein willkommener Anla"s f"ur Abenteuer jenseits
der klassischen Auftragssituation, und die Bezahlung mit magischen
Gegenst"anden ist ein Weg, ein eventuell au"ser Kontrolle geratenes
Arsenal der SpF an magischen Gegenst"anden wieder auszud"unnen und den
SpF dabei gleichzeitig ad"aquate Gegenleistung zu bieten.

\fbox{\parbox{0.98\textwidth}{%
  \textbf{Beispiel:} \emph{\small Der Priester Fares lebt in klosterartiger
    Tempelgemeinschaft. Die Mitglieder der Gemeinschaft haben kein
    wesentliches Privateigentum. Nat"urlich verlangt die
    Tempelgemeinschaft von ihren Mitgliedern auch kein Geld f"ur das
    Leben im Tempel und das Lernen von Wundertaten. Wovon sollten die
    Priester das auch bezahlen? Andererseits stellt Fares nat"urlich
    seine Erl"ose aus den Abenteuern der Tempelgemeinschaft zur
    Verf"ugung. Und die Tempelgemeinschaft wiederum bezahlt manchmal
    auch Lehrmeister, die Fares F"ahigkeiten beibringen, die f"ur seine
    Dienste im Auftrage der Gottheit unerl"a"slich sind. (Die
    Bezahlung k"onnte z.\,B. so aussehen, da"s der entsprechende
    Lehrmeister im Tempel von seiner Gicht geheilt wird.) Fares ist
    als tiefgl"aubiger Priester von seiner Tempelgemeinschaft
    "uberzeugt und w"urde die beschriebenen spieltechnischen
    Privilegien niemals egoistisch ausnutzen. (Auch das Einsetzen von
    nur dem Mindestdrittel an EP k"onnte man als schamloses Ausnutzen
    der Tempelgemeinschaft interpretieren, denn es entspricht einem
    minimalen pers"onlichen Einsatz beim Lernen.) Mit Spielern, die
    ihre Rolle glaubw"urdig spielen, sind derartige
    flexible Regelungen beim Lernen ohne Probleme m"oglich. Spieler,
    die solche gutm"utigen Regelungen massiv zum eigenen
    spieltechnischen Vorteil auszunutzen, m"ussen sich fragen lassen,
    ob sie geeignet sind, einen Priester zu spielen.}
}}

\fbox{\parbox{0.98\textwidth}{%
  \textbf{Beispiel:} \emph{\small Oxymanos, der
    Magier, m"ochte einen m"achtigen Zauber lernen, den ihm niemand aus
    seiner Gilde in Chryseia beibringen kann. Nach einigen Recherchen findet er
    heraus, da"s es in Candranor einen Lehrmeister gibt, der diesen
    Zauber beherrscht. Da Oxymanos' Abenteurergruppe auch
    daran interessiert ist, da"s 
    er den Zauber lernt (und er verspricht, bei sp"ateren
    Gelegenheiten sich erkenntlich zu zeigen), begleiten sie ihn nach 
    Candranor. Dort angekommen, stellt sich heraus, da"s der magische
    Lehrmeister von seiner letzten Expedition in die tegarische Steppe nicht
    zur"uckgekehrt ist und als verschollen gilt. Die Gruppe bricht
    auf, ihn zu finden. Nach einem l"angeren Abenteuer gelingt es
    ihnen, den Magier aus der Gefangenschaft eines Tegarenstammes zu
    befreien. Dankbar erkl"art er sich bereit, Oxymanos kostenlos den
    begehrten Zauberspruch zu lehren.}
}}

\subsubsection{Unterbrechungen des Lernens}

Im Gegensatz zu den Standardregeln sind Unterbrechungen des
Lernvorgangs grundsätzlich schadlos möglich. Auch kann ein Lernvorgang
beliebig modularisiert werden in Teile, die durch Unterweisung, durch
Selbststudium oder durch Praxis gelernt werden. Diese Regelung sollte
natürlich nicht exzessiv ausgenutzt werden und kann vom SpL
ausgesetzt werden, wo dies sinnvoll erscheint.  Aber gerade bei sehr
umfangreichen Lernvorgängen (z.\,B. die sehr aufwendige Steigerung
eines bereits hohen Erfolgswertes für eine Waffe) erscheint es
un\-glaub\-wür\-dig, daß dies in einem Stück und an einem Ort zu erfolgen
hat.

\subsubsection{Lernen durch Praxis}

Das Lernen durch Praxis erfolgt exakt nach Standardregelwerk
[DFR\,284]. Insbesondere sind die Lebenshaltungskosten in der Tat
nicht in den Lernkosten enthalten (vgl. weiter unten). Die
Lebenshaltungskosten fallen bei einem Lerntempo von 500\,FP pro Tag
auch kaum ins Gewicht. Spesen fallen ohnehin nicht an.

\subsubsection{Lerndauer}

Die \textbf{Lerndauer} von einem Tag pro 10\,FP (bzw. pro 5\,FP bei
Lernen durch Selbststudium bzw. mit weniger als 1/3 EP (s.\,u.)) wird
nur erreicht, wenn der Lernende sich ausschlie"slich dem Lernen widmet
und dabei \textbf{optimale Bedingungen} vorfindet. Nur unter sehr
speziellen Umst"anden ist es denkbar, da"s sich die Lernzeit weiter
verk"urzt. Viel eher wird sich die Lernzeit durch widrige Umst"ande
verl"angern.

Die Dauer des Steigern des AP-Maximums [DFR\,288] wird im Normalfall ebenfalls
wie oben berechnet. Es gibt aber eine spezielle Art des Selbststudiums, die im
Abschnitt \ref{selbststudium} beschrieben wird und, wie im Regelwerk
beschrieben, immer einen Monat dauert.

\subsubsection{Zusammensetzung der Lernkosten}

Die Lernkosten beinhalten (im Gegensatz zu den Standardregeln) den
\textbf{Lebensunterhalt} w"ahrend der Lernzeit (wobei die Kosten für den
Lebensunterhalt sich mit einem eigenen Haus nur unwesentlich reduzieren).
Dabei ist ber"ucksichtigt, da"s ein Abenteurer nach langem anstregenden, aber
erfolgreichen Abenteuer vermutlich nicht in Armut leben, sondern vielmehr eine
\textbf{"uberdurchschnittliche Lebensqualit"at} genie"sen will und da"s es
einigen Aufwand erfordert, die oben beschriebenen \textbf{optimalen
  Bedingungen} zu erreichen. Der SpF bleibt keine Zeit, sich um die
Nahrungsbeschaffung oder die Instandhaltung ihres Haushaltes zu k"ummern.
Dazu mu"s sie bezahlte Dienstleistungen in Anspruch nehmen. Die
Lebenshaltungskosten lassen sich reduzieren, wenn sie derartige Aufgaben selber
wahrnimmt, was dann allerdings die Lerndauer erh"oht (und dadurch nat"urlich
die Ersparnis an Lebenshaltungskosten wieder zunichte machen kann, weil f"ur
das Erlernen einer bestimmten F"ahigkeit der Lebensunterhalt nun f"ur eine
l"angere Zeit zu bestreiten ist).

Die Lebenshaltungskosten sind im Mindestdrittel der Lernkosten, das in
Geld zu erbringen ist, enthalten. Dieses Mindestdrittel ist (entgegen
den Standardregeln) auch im Selbststudium nötig. Außer den
Lebenshaltungskosten enthält es Kosten für Lehrmittel, sonstige Spesen
und Nebenkosten (Reisekosten zu Studienstätten, Bibliotheksbenutzung
etc.) und einen gewissen Sockelbetrag für den Lehrmeister (der im
Selbststudium durch Mehrkosten für Lehrmittel, Spesen und (bedingt durch
die längere Lerndauer) Lebenshaltung ersetzt wird).

Das zweite Drittel der Lernkosten kann alternativ durch eigene
Erfahrung (EP) oder durch die Leistung eines besseren Lehrmeisters
aufgebracht werden (der dann entsprechend mehr Geld verlangt).

Das letzte Drittel wird Grundsätzlich immer durch Einbringen der
eigenen Erfahrung, also in EP bezahlt. Nur bei exzellenten
Lehrmeistern ist es möglich, auch dieses letzte Drittel durch
Bezahlung des Lehrmeisters, also durch Geld zu ersetzen.

Tabelle~\ref{tab:lernkosten} soll veranschaulichen, wie sich die Lernkosten
aufteilen und unter welchen Umst"anden eine Reduzierung der Kosten
m"oglich sein mag.

\begin{table}[htbp]
  \begin{center}

    {\footnotesize 
      \begin{tabular}{|p{1cm}|c|p{5cm}|p{7.5cm}|}
        \hline
        & \textbf{Zahlungsart}& \textbf{Bedeutung}
        & \textbf{Kostenbefreiung oder -reduzierung,
          Anmerkungen} \\
        \hline
        erstes& EP & nicht m"oglich& --- \\
        \cline{2-4}
        Drittel & Geld & Lebensunterhalt (Kost und Logis etc.), Lehrmittel,
        ggf. Reisekosten, jegliche sonstige Nebenkosten, bei
        Unterweisung Grundbetrag für den Lehrmeister & Verbilligte
        Unterkunft oder Verpflegung werden gew"ahrt, Lehrmittel werden zur
        Verf"ugung gestellt, SpF nimmt verringerte Lebensqualit"at hin,
        Lehrmeister stellt seine Dienste verbilligt
        oder kostenlos zur Verf"ugung etc.\\
        \hline
        zweites Drittel & EP & Einbringen der eigenen Erfahrung & Auch wenn
        $\frac{\sf 2}{\sf 3}$ der Lernkosten in EP bezahlt werden, ist zumindestens ein 
        unerfahrener Lehrmeister erforderlich. Ausnahme:
        Selbststudium, was allerdings zusätzliche Hilfsmittel (spezielle
        Literatur etc.)
        erfordert und doppelt so lange dauert.\\
        \cline{2-4}
        & Geld & Bezahlung des Lehrmeisters & Lehrmeister stellt seine Dienste verbilligt
        oder kostenlos zur Verf"ugung. \\
        \hline
        drittes & EP &  Einbringen der eigenen Erfahrung & ---\\
        \cline{2-4}
        Drittel & Geld & Bezahlung des Lehrmeisters, nur m"oglich bei
        exzellentem Lehrmeister, dabei ist unter Umst"anden eine erh"ohte
        Bezahlung oder Lernzeit erforderlich. & Lehrmeister stellt seine Dienste verbilligt
        oder kostenlos zur Verf"ugung. \\
        \hline
      \end{tabular}
      }

    \caption{Aufteilung der Lernkosten bei Selbststudium und Unterweisung}
    \label{tab:lernkosten}
  \end{center}
\end{table}

\subsubsection{Lehrmeister}

Im Gegensatz zu den Standardregeln können durchaus auch SpF als
Lehrmeister tätig werden. Es ist grunds"atzlich nur m"oglich,
F"ahigkeiten zu lehren, die man selber beherrscht. Maximal kann der
Erfolgswert oder die Stufe vermittelt werden, die man selber
beherrscht.  AP-Steigerungen k"onnen nur bis zu dem Grad vermittelt
werden, f"ur den man selber bereits die AP-Steigerung mindestens
einmal vollzogen hat. SpF gelten grunds"atzlich als
\textbf{unerfahrene} Lehrmeister, d.\,h. der Lernende mu"s $\frac{\sf
  2}{\sf 3}$ der Lernkosten in EP zahlen. Das restliche Drittel muß
weiterhin mit Geld bezahlt werden, weil sich dieses Drittel ohnehin
hauptsächlich aus Lebensunterhalt und Nebenkosten zusammensetzt. Der
geringe Anteil, der davon für den Lehrmeister vorgesehen ist, wird
einerseits von den Lebenshaltungskosten der lehrenden SpF aufgezehrt,
andererseits durch die bei einem unerfahrenen Lehrmeister erhöhte
Lernzeit (die wiederum die Lebenshaltungskosten erhöht).

Zum \textbf{Standardlehrmeister} im Sinne des Regelwerks (bei dem der
Lernende nur $\frac{\sf 1}{\sf 3}$ der Lernkosten in EP zahlen mu"s)
wird man, wenn man eine bestimmte F"ahigkeit (bis zum zu lehrenden
Erfolgswert bzw. zur lehrenden Stufe) \textbf{zweimal} gelernt hat. Um
zum \textbf{exzellenten} Lehrmeisters zu werden (bei dem die gesamten
Lernkosten mit Geld bezahlt werden können), mu"s man die zu lehrende
F"ahigkeit \textbf{viermal} gelernt haben. NSpF, die eine F"ahigkeit
als Teil ihres Alltags st"andig auf professionellem Niveau
durchf"uhren, gelten grunds"atzlich als Standardlehrmeister. Bei
besonderer Begabung k"onnen sie auch exzellente Lehrmeister sein.

\subsubsection{Selbststudium}
\label{selbststudium}

Das Selbststudium ist gewisserma"sen Forschungsarbeit. Das Wissen um
die zu erlernenden F"ahigkeiten mu"s erst zusammengetragen werden
(durch Literaturstudium, Ausprobieren, Feldstudien oder andere
denkbare Forschungsmethoden). Das Selbststudium dauert somit, wie im
Regelwerk festgelegt, deutlich l"anger als das Lernen von einem
Lehrmeister. Dies gilt insbesondere f"ur SpF, die keine so gro"se
Erfahrung im Selbststudium haben wie NSpF, die sich weitestgehend auf
eine Disziplin spezialisiert haben und sich im wesentlichen mit nichts
anderem mehr besch"aftigen (und nat"urlich auch nur in ihrer Disziplin
die Methoden des Selbststudiums beherrschen).

Wie schon gesagt, erfordert das Selbststudium im Gegensatz zum
Standardregelwerk das Mindestdrittel in Geld. Dafür kann man aber auch
neue Fertigkeiten/Zauberkünste durch Selbststudium erlernen.

Eine Ausnahme stellt die Steigerung des AP-Maximums dar. Da für eine
SpF gewissermaßen das "`ganz normale Abenteurerleben"' das Training
zur Steigerung der Ausdauer darstellt, kann sie das AP-Maximum
"`nebenbei"' während des Abenteuers, einer Lernphase oder wann auch
immer steigern. Dies erfordert immer \textbf{1~Monat} (währenddessen
das "`normale Leben"' weitergeht, es kann also parallel "`normal"'
gelernt werden). Es müssen die \textbf{gesamten FP-Kosten mit EP}
beglichen werden. Der Lebensunterhalt wird in diesem Falle gesondert
abgerechnet (wobei allerdings keine besonderen Maßnahmen zu ergreifen
sind und dieses Selbststudium auch unter bescheidenen
Lebensverhältnissen möglich ist -- und wenn parallel "`normal"'
gelernt wird, sind die Lebenshaltungskosten ja eh schon
abgegolten). Bis auf die Beschränkung, daß die gesamten FP-Kosten mit
EP bezahlt werden müssen, und der Möglichkeit, parallel andere Dinge
zu tun, entspricht dies also der AP-Steigerung, wie sie ursprünglich
im Regelwerk vorgesehen ist. (Wenn man das AP-Maximum nach dem
normalen Verteilungsschema mit Lehrmeister steigert, gelten aus
Konsistenzgründen die normalen Lernzeiten.)

\subsubsection{Lernen im Alltagsleben}

Manchmal läßt das nächste Abenteuer auf sich warten, und dann gehen
selbst gestandene Helden für längere Zeit einem ganz normalen Alltag
nach, bleiben seßhaft an einem Ort, verdienen ihren Lebensunterhalt in
einem "`anständigen"' Beruf usw. In einer solchen Situation ist das
Lernen mit maximaler Geschwindigkeit und minimaler Ablenkung nicht
unbedingt sinnvoll. Hält sich eine SpF für Monate oder gar Jahre in
einer angestammten Umgebung auf, mit festem Wohnsitz, einem
auskömmlichen Beruf oder anderen Einnahmequellen, einem
funktionierendem sozialen Netzwerk und dergleichen mehr, dann kann sie
dabei --~gewissermaßen als Teil ihres Alltags~-- mit verminderter
Geschwindigkeit lernen. Regeltechnisch wird dies so simuliert, daß
angenommen wird, daß die SpF das Mindestdrittel der Lernkosten in Geld
aus ihren regelmäßigen Einnahmen bestreitet. Dabei sinkt die
Lerngeschwindigkeit auf \textbf{ein Viertel}, also \textbf{2,5 FP pro
  Tag}. Explizit bezahlt werden also nur noch die übrigen zwei Drittel
der Lernkosten, und zwar zur einen Hälfte zwingend aus EP, zur anderen
Hälfte aus einer beliebigen Mischung von EP und Geld (bei Lernen
durch Unterweisung) bzw. ebenfalls durch EP (bei Selbststudium). Einen
Zeitunterschied zwischen Selbststudium und Lernen durch Unterweisung
gibt es auf dieser Abstraktionsebene nicht mehr.

\subsubsection{Wie lernen NSpF?}

NSpF ziehen nicht unbedingt auf Abenteuer aus, um EP zu sammeln. Wie
lernen sie?  Zum einen gibt es die M"oglichkeit, bei einem
\textbf{exzellenten} Lehrmeister die Lernkosten \textbf{komplett in
  Geld} zu bezahlen (vgl.  Tabelle~\ref{tab:lernkosten}), zum anderen
eine spezielle Form des Selbststudiums: Spezialisierte NSpF haben
m"oglicherweise einen Weg gefunden, EP durch ihre Forschungen und
"Ubungen zu generieren. SpF steht dieser Weg nur in Ausnahmef"allen
offen, wenn sie sich nicht weitgehend vom Abenteurerleben
verabschieden und stattdessen der geregelten Spezialt"atigkeit eines
Gelehrten, eines se"shaften Magiers, eines Kampftrainers etc.
nachgehen wollen.  Das hier beschriebene Selbststudium ist "ubrigens
auch der spieltechnische Weg, auf dem neuartige Fertigkeiten,
Zauberspr"uche usw. entwickelt werden.



%Eine allgemeinere M"oglichkeit, das Lehren von F"ahigkeiten zu
%erlernen, ist es, die Fertigkeit \emph{Lehren}
%(Tabelle~\ref{tab:lehren}) zu erlernen. Sie mu"s
%jeweils getrennt f"ur "`ZEP"'-, "`AEP"'- und "`KEP"'-F"ahigkeiten
%gelernt werden (vgl. T12a).
%
%\begin{table}[htbp]
%
%  \begin{center}
%    \framebox{
%      \begin{minipage}[c]{16cm}
%
%        \begin{description}
%        \item[\large Lehren] ("`AEP"' kann nur mit AEP gelernt werden, "`KEP"'
%          zus"atzlich mit KEP, "`ZEP"' zus"atzlich mit ZEP)
%        \item[Grundf"ahigkeit:] Magister; Z (nur "`ZEP"'); Kr, S"o (nur
%          "`KEP"')
%        \item[Ausnahmef"ahigkeit:] Kr, S"o (nur "`ZEP"'); Z (nur "`KEP"')
%        \end{description}
%        \begin{tabular}{cccl}
%          \textbf{Stufe} & \textbf{Anteil} & \textbf{Kosten} & \\
%          \vspace{1mm}
%          1 & $\frac{\sf 1}{\sf 9}$ & 250\,FP & \\
%          \vspace{1mm}
%          2 & $\frac{\sf 2}{\sf 9}$ & 500\,FP & \\
%          \vspace{1mm}
%          3 & $\frac{\sf 1}{\sf 3}$ &1000\,FP & entspricht dem Standardlehrmeister aus dem Regelwerk\\
%          \vspace{1mm}
%          4 & $\frac{\sf 4}{\sf 9}$ &2000\,FP & \\
%          \vspace{1mm}
%          5 & $\frac{\sf 5}{\sf 9}$ &4000\,FP & \\
%          \vspace{1mm}
%          6 & $\frac{\sf 2}{\sf 3}$ &8000\,FP & entspricht dem exzellenten Lehrmeister\\
%        \end{tabular}
%        \vspace*{5mm}
%
%        {\small
%          \textbf{Anteil} ist der Anteil an FP, der beim Lernen durch den
%          Lehrmeister geleistet wird (vgl. Tabelle \ref{tab:lernkosten}).
%          
%          }
%      \end{minipage}
%      }
%    \caption{Fertigkeit \emph{Lehren}}
%    \label{tab:lehren}
%  \end{center}
%\end{table}


\subsection{Schicksalsgunst}

Gemäß Korrektur in der 2. Auflage von [DFR\,291] (die aber nicht in
den Errata erwähnt wird) kann Schicksalsgunst auch zur Wiederholung
eines gegen die SpF gerichteten \textbf{EW:Meucheln} genutzt werden.
(Das Beispiel auf [DFR\,161] legte das auch schon in der 1. Auflage
nahe.)

Wird Schicksalsgunst benutzt, um die EW für einen gezielten Schuß zu
wiederholen, werden immer \textbf{beide} EW (nach Einsatz
\textbf{eines} Punktes SG) wiederholt.

\section{Aktionsphasen}

\subsection{Bewegung}

Figuren, die sich in der vorangegangenen Runde \textbf{vom Gegner gel"ost} 
haben, z"ahlen noch nicht als \textbf{fliehende} Figuren, d"urfen sich 
also nicht automatisch als erste bewegen.

Bewegungen auf dem Spielfeldraster sind auch \textbf{diagonal}
m"oglich. Solch ein Schritt z"ahlt wie die Bewegung "uber
\textbf{1,5~Felder}. Wenn dieser Schritt aber die einzige Bewegung
einer SpF ist, dann zählt dies im Sinne der Regeln immer noch als eine
Bewegung von nur 1\,m.

Während der Bewegung sind \textbf{Drehungen} im Prinzip
uneingeschränkt möglich. Die separate Drehung um bis zu 90$^\circ$
nach Abschluß aller Bewegungen [DFR\,87] wird in umgekehrter
Initiativreihenfolge durchgeführt (also die Partie, die die Initiative
gewonnen hat, dreht sich zuletzt).

\subsection{Zeitabfolge in Aktionsphasen}
\label{timing}

Solange Aktionsphasen wie in [DFR\,86ff] beschrieben \textbf{rundenweise}
abgehandelt werden, gelten folgende Anmerkungen:

Die Entscheidung, welche Handlungen die Figuren in einer Runde
ausführen, wird (zumindestens im Prinzip) \textbf{gleichzeitig} nach
Abschluß der Bewegungen getroffen. (In der Praxis ist es am
einfachsten, wenn der SpL zunächst heimlich die Handlungen für
die NSpF festlegt, woraufhin dann die Spieler für ihre Figuren
entscheiden. Sollte es in seltenen Fällen für einen "`realistischen"'
Spielablauf nötig sein, kann man die Aktionen auch geheim auf Zetteln
notieren und dann gleichzeitig offenlegen.)

Als Teil der Handlungsentscheidung kann eine Figur auch freiwillig
ihren Handlungsrang so reduzieren, daß sie mit ihrer Handlung zuletzt
an die Reihe kommt. Wollen das mehrere Figuren, kommen sie alle am
Ende der Runde in \textbf{umgekehrter} Gw-Reihenfolge an die Reihe
(höchste Gw zuletzt).

Die so festgelegten Handlungen werden dann in \textbf{Handlungs-
  bzw. Angriffsrangreihenfolge durchgef"uhrt}. Eine festgelegte
Handlung kann im Nachhinein \textbf{nicht mehr ge"andert} werden (mit
der Ausnahme einiger spezieller Situationen, in denen die Regeln dies
explizit erlauben). Allerdings ist es m"oglich, auf die Ausübung der
angesagten Handlung zu \textbf{verzichten}, wenn man an der Reihe
ist. (Manchmal hat man auch gar keine andere Wahl\dots) Außerdem ist
zu beachten, daß prinzipiell nur die \textbf{Art der Handlung} vorab
ausgewählt werden muß (was zumeist der Auswahl einer der mit
Buchstaben versehenen Optionen in [DFR\,220f] entspricht). Details
können dann spontan und durchaus in Reaktion auf das zuvor Geschehene
gewählt werden (z.\,B. das Ziel eines Angriffs, die zu ziehende Waffe
oder Auswahl und Ziel eines Zaubers).
% Letzteres folgt einer JEF-Äußerung, daß beim Fechten das Ziel des
% zweiten Angriffs spontan gewechselt werden kann, z.B. wenn das
% ursprüngliche Ziel bereits kampfunfähig ist.
Hat eine Figur sich aber z.\,B. auf einen Angriff festgelegt, wird
dann aber \textbf{zu Fall gebracht} (vgl. [DFR\,88]),
\textbf{entwaffnet} oder im Handgemenge \textbf{festgehalten}, bevor
sie mit ihrer Handlung an der Reihe ist, dann entfällt ihr Angriff und
damit ihre Handlung ganz. Sie kann sich nicht spontan entscheiden,
statt des Angriffs \textbf{aufzustehen}, die \textbf{Waffe aufzuheben}
bzw. sich \textbf{loszureißen}. Diese Handlungen können erst in der
nächsten Runde gewählt werden. Eine Figur kann auch nicht
"`präventiv"' eine Handlung wählen, für die die Voraussetzungen nach
Abschluß der Bewegungen noch gar nicht vorliegen (z.\,B. eine Waffe
aufzuheben für den Fall, daß sie zu Boden fallen sollte, bevor die
Figur mit ihrer Handlung an die Reihe kommt).

\subsubsection{Andauernde Effekte}

Wird durch eine Handlung ein Effekt ausgelöst, der eine Runde
bzw. 10sec dauert und andere Figuren betrifft, so wirkt er sich auf
die jeweils nächste Handlungs- und Bewegungsmöglichkeit jeder
betroffenen Figur aus, egal in welcher Reihenfolge sie eintreten. Am
besten ist dies für den Zauber \emph{Verwirren} bei M5 beschrieben,
gilt aber analog für andere Effekte wie Bewußtlosigkeit durch einen
Fausthieb oder Schock durch einen Peitschenhieb.

Mißlingt also z.\,B. einer Figur nach LP-Verlust durch einen
Peitschenhieb ihr PW:Wk, nachdem sie mit ihrer Handlung an der Reihe
war, verliert sie die Bewegungs- und Handlungsmöglichkeit der nächsten
Runde. Findet der Peitschenhieb aber noch vor ihrer
Handlungsmöglichkeit statt, verliert sie die für die aktuelle Runde
geplante Handlung und die Bewegung der folgenden Runde. Sie kann aber
in der folgenden Runde schon wieder normal handeln.

Für länger anhaltende Effekte gilt Entsprechendes, z.\,B. führt 1min
Bewußtlosigkeit zum Verlust der nächsten 6 Bewegungs- und
Handlungsmöglichkeiten.

Tritt der Spezialfall ein, daß eine Figur mehrere
Handlungsmöglichkeiten hat, z.\,B. durch \emph{Beschleunigen}, wird
der Ablauf so gut es geht im obigen Sinne geregelt, wobei im
Zweifelsfall zu Ungunsten der Figur „gerundet“ wird. Ein
beschleunigter Kämpfer, der nach seinem ersten Angriff Opfer von
\emph{Verwirren} wird, verliert den zweiten Angriff, die Bewegung der
folgenden Runde und den ersten Angriff der nächsten Runde, kann aber
den zweiten Angriff der folgenden Runde wieder durchführen. Wirkt
\emph{Beschleunigen} aber in der folgenden Runde schon nicht mehr oder
will die Figur eine Handlung ausführen, auf die sich
\emph{Beschleunigen} nicht auswirkt, verliert sie diese
Handlungsmöglichkeit komplett.

Komplizierter wird es, wenn ein Effekt \textbf{Wehrlosigkeit} auslöst,
denn diese wirkt sich auf die Handlungen aller anderen angreifenden
Figuren aus, die zu ganz verschiedenen Zeitpunkten mit ihrer Handlung
an die Reihe bekommen. Wenn nicht ausdrücklich etwas anderes geregelt
ist, gilt eine Wehrlosigkeit von einer Runde bzw. 10sec von dem
Zeitpunkt an, zu dem der Effekt einsetzt, bis zum Ende der folgenden
Runde.

Wann immer die Regeln sich ausdrücklich auf die aktuelle bzw. die
folgende Runde beziehen, gilt dies weiterhin uneingeschränkt.

\subsubsection{Sekundengenaue Simulation}

Obige Anmerkungen unterstreichen die Wichtigkeit von Angriffs-
bzw. Handlungsrang (und damit von Gw bzw. deren Reduzierung durch
schwere Rüstungen). Bei \textbf{sekundengenau} simulierten
Handlungsabläufen gelten diese Anmerkungen nicht mehr. Die Gw spielt
dann aber durch die jetzt erfolgenden PW eine wichtige Rolle. Generell
gilt aber ohnehin, das sekundengenau simulierte Handlungsabläufe die
Ausnahme darstellen. Dem SpL sollte klar sein, daß sie ihm
üblicherweise sehr viel Regelimprovisation abfordern werden. Häufig
läßt sich mit ähnlich viel Regelimprovisation (aber weniger Aufwand)
eine Situation auch rundenweise befriedigend abhandeln.

\section{Kampf}

\subsection{Nahkampf}

Gemäß [DFR\,223f] ist die Anzahl an Gegnern, die eine Figur
gleichzeitig im Nahkampf angreifen können, durch die Anzahl freier
Kanten des Feldes (oder der Felder), das (bzw. die) sie einnimmt,
bestimmt. Welche Gegner dürfen nun aber angreifen, wenn mehr als die
Maximalzahl um die Figur herumstehen? Nur nach Angriffsrang zu gehen,
könnte zu der Situation führen, daß zahlreiche schnelle Angriffe auf
das eine Ende einer Riesenschlange die späteren Angriffe auf das
andere Ende unterbinden. Für diesen Fall hier also verfeinerte Regeln:

Ein Feld verfügt über acht sogenannte \textbf{Angriffswege}, nämlich
je einen über jede Ecke und je einen über jede Kante. Der Angriffsweg
über eine Kante ist benachbart zu den beiden Angriffswegen über die
benachbarten Ecken, und der Angriffsweg über eine Ecke ist benachbart
zu den beiden Angriffswegen über die benachbarten Kanten. (Für größere
Figuren gilt entsprechendes für die außen liegenden Kanten und Ecken
der eingenommenen Felder.) Verfügen die Felder zweier Figuren über
eine gemeinsame Kante, dann erfolgen Angriffe über diese Kante oder
eine der beiden benachbarten Ecken. Sind die beiden Figuren diagonal
benachbart, erfolgt der Angriff über die gemeinsame Ecke oder eine der
beiden benachbarten Kanten des Feldes des Angriffszieles. Bei weiteren
Angriffen gilt, daß der gewählte Angriffsweg nicht unmittelbar
benachbart zu einem bereits genutzten Angriffsweg sein
darf. \textbf{Lange Spießwaffen} können jeden Angriffsweg wählen, der
noch nicht benutzt wird. Weder behindern sie benachbarten
Angriffswege, noch werden sie von benutzten benachbarten Angriffswegen
behindert. \textbf{Zweihändige Hiebwaffen} beanspruchen immer den
zentralen und beide benachbarten Angriffswege.

\subsection{Fernkampf}

Schußwaffen können, wenn nicht ausdrücklich etwas anderes angegeben
wird, jede Runde abgefeuert werden. Die Geschosse müssen dabei in einem
geeigneten Behälter am Körper bereitgehalten werden, so daß schnell
genug "`nachgeladen"' werden kann.

Bei Wurfwaffen ist die Situation weniger klar, weil ja die gesamte
Waffe weggeworfen wird. Prinzipiell erfordert das Ziehen einer neuen
Wurfwaffe eine ganz Runde, bevor diese wieder zum Einsatz gebracht
werden kann. Je nach Waffenart können aber Waffen günstig genug
bereitgehalten werden, so daß sie jede Runde geworfen werden
können. Bei Wurfstern und Wurfscheibe geht aus der Beschreibung
[DFR\,210] hervor, daß sogar mehrere dieser Waffen innerhalb einer
Kampfrunde geworfen werden können. Beim Wurfspeer erwähnt die
Beschreibung, daß drei Speere in der Hand bereitgehalten werden
können. Beim schwereren Wurfspieß ist dies offenbar nicht
möglich. Andere kleinere Wurfwaffen (Wurfmesser, Wurfpfeile, \dots)
können je nach Situation (SpL-Entscheidung) bereitgelegt oder
in der freien Hand bereitgehalten werden.

\subsection{Handgemenge}

Man beachte, daß eine im Handgemenge \textbf{Festgehaltener} zwar
"`weitgehend handlungsunfähig"' ist [DFR\,239], dies aber nichts über
ihre Fähigkeit zur \textbf{Abwehr} aussagt (was ja formal keine Handlung
ist, vgl. Anhang \ref{behindert}). Das Beispiel in [DFR\,240] legt
aber nahe, daß es für Dritte möglich ist, den Festgehaltenen zu
fesseln. Eine echte Verknotung würde mindestens 1min dauern (oder
sogar 10min mit \emph{HoJoJutsu} [KTP\,161]). Hier geht es aber wohl
nur darum, das Opfer so mit Seilen zu umwickeln, daß das weitere
Festhalten dann automatisch gelingt und die Seile in Ruhe verknotet
werden können. Um dies zu simulieren, darf einmal pro Runde jeder, der
nicht selbst mit dem Festhalten beschäftigt ist, als Handlung
versuchen, den Festgehaltenen wirkungsvoll mit einem Seil zu
umwickeln. Dazu muß ihm ein \textbf{EW:Seilkunst} gelingen, bei dem
kritische Fehler und Erfolge keine besonderen Auswirkungen
haben. Gelingt dem Festgehaltenen das Lossreißen, scheitern alle
Umwicklungsversuche dieser Runde automatisch.

Der \textbf{Rüstungsbonus} findet auch im Handgemenge Anwendung (im
Gegensatz zum Waffenrang).

\subsection{Rüstung und Verteidigung}

Im Sinne von [DFR\,94] gelten Angriffe als \textbf{von vorne}, wenn
sie von den drei direkt vor einer Figur gelegenen Feldern ausgeführt
werden. \textbf{Von der Seite} erfolgen Angriffe, die weder von vorne
noch von hinten erfolgen.

Figuren, die mit LR oder schwerer gerüstet sind, erleiden
spieltechnisch kaum weitere Nachteile, wenn sie auch Lederarm- und
-beinschienen und einen ledernen Halsschutz tragen [DFR\,245]. Auch
NSpF, die auf einen Kampf eingestellt sind, werden nach Maßgabe des
SpL diese Rüstungsteile tragen, selbst wenn dies nicht ausdrücklich in
den Spieldaten verzeichnet ist.

Für Belange des \textbf{Kampfes ohne Waffen} gilt das Tragen eines
Helmes, eines Halschutzes oder von Arm- oder Beinschienen aus Leder
(auch einzeln) wie das Tragen einer LR. Das Tragen eines
entsprechenden Rüstungsteils aus Metall gilt wie das Tragen von PR
(wobei der Gegner eines Faustkämpfers weiterhein eine vollwertige PR
oder besser benötigt, um gegen den Faustkampf immun zu sein).

Die Schwachstellen der kanthanischen Rüstungen [KTP\,254] lassen
sich in der Tat nur durch \textbf{gezielte Hiebe}, nicht aber durch
normale \textbf{kritische Treffer} ausnutzen. Bei gezielten Hieben auf
die Gliedmaßen schützen die kanthanischen Rüstungen grundsätzlich nur
wie \textbf{TR} (so die jeweilige Rüstung überhaupt den betroffenen
Körperteil abdeckt). Der gezielte Hieb auf die Schwachstelle am Rumpf
erfolgt mit \textbf{WM+4} auf den 2. WW:Abwehr, hat aber keinen
kritischen Schaden zur Folge, sondern lediglich den Effekt, daß die RK
nur TR ist. Ein \textbf{kritischer Erfolg} hat allerdings den Effekt
von \textbf{21--35} auf \textbf{Tabelle 4.5}. (Die Schwachstelle der
Rüstung erlaubt kein Zielen auf lebenswichtige Organe. In der Regel
ist es effektiver, die Schwachstellen an den Gliedmaßen auszunutzen.)

% Folgender Absatz gestrichen wegen eindeutiger Aussage JEFs.
% 
% Die Beschränkung des Rüstungsschutzes gegen Armbrüste, Langbögen und
% Kompositbögen [DFR\,97] gilt nur bei Metallplatten (die durch derartige
% Geschosse durchschlagen werden). Bei ausreichend anderer Art der Rüstung
% (z.\,B. bei manchen natürlichen Rüstungen) gilt auch bei diesen Schußwaffen
% der normale Rüstungsschutz.

\subsection{Erfolgswert innerhalb einer Waffengattung}

Innerhalb einer Gruppe von Waffen, die mit denselben
\textbf{Grundkenntnissen} gelernt werden, greift eine SpF mindestens
mit dem um vier reduzierten Erfolgswert der am besten beherrschten
Waffe der Gruppe an (in Anlehnung an M2 und Annäherung an
M5). Explizit gelernt wird eine Waffe allerdings wie üblich vom
Startwert +4 beginnend.

\subsection{Improvisierte Waffen}

Eine Zauberölamphore muß \textbf{schwer} treffen, um den Gegner mit
Zauberöl zu benetzen.  Bei einem leichten Treffen wird nur das
\textbf{Feld} benetzt (also der Fußboden), auf dem der Gegner steht.
Ein "`Zauberöl-Molli"' wäre sonst zu gefährlich. Selbst wenn man
zugesteht, daß so ein Molli sich nur mit einer gewissen (aber durchaus
hohen) Wahrscheinlichkeit beim Aufprall entzündet, wäre die
Schadenswirkung in Anbetracht der Tatsache, daß faktisch keine Abwehr
möglich ist, zu kraß. Außerdem stellt die Abwehr eines
Fernkampfangriffes u.\,a. das Ausweichen bzw. Ablenken mit dem Schild
dar, wobei fraglich ist, wie es dabei zu einer Benetzung des
Abwehrenden mit Zauberöl kommen kann. [DFR\,253f]

\subsection{Auswirkungen und Heilung von erlittenem Schaden}

Die Nachteile, die eine Figur erleidet, wenn sie mindestens die Hälfte
ihrer LP verloren hat bzw. nur noch über 3 oder weniger LP verfügt,
gelten grundsätzlich nur für \textbf{Lebewesen}. Insbesondere Untote
leiden nicht unter diesen Nachteilen und würfeln auch nicht auf
Tabelle~2.5. (Siehe [MDS\,85] und eine offizielle Klarstellung im
Midgard-Forum.) Für andere unbelebte oder magisch belebte Figuren gilt
nach Maßgabe des SpL Entsprechendes. Auch die Interpretation
der Auswirkungen kritischen Schadens bei Figuren, die keine normalen
Lebewesen sind, muß entsprechend angepaßt werden. (Beispielsweise wird
eine Skelett durch eine Augen- oder Wirbelsäulenverletzung in keinster
Weise beeinträchtigt. Allerdings hat auch ein Skelett einen kritischen
Punkt, der es beim Würfeln einer 100 einfach zusammenklappen läßt.)

\subsubsection{AP-Verlust}

Eine Figur mit \textbf{0\,AP} [DFR\,101] erh"alt \textbf{WM--4} nicht
nur auf EW:Angriff, sondern auch auf alle EW und WW, die mit
\textbf{starker k"orperlicher Anstrengung} verbunden sind. WW, bei
denen die dahinterstehende Handlung einer Abwehr gleicht, können
auch ganz verfallen.

Eine lebensgef"ahrlich verwundete Figur (die also i.\,d.\,R. 3 oder
weniger LP hat) verliert automatisch \textbf{alle AP} und regeneriert
keine AP, solange sie als lebensgef"ahrlich verwundet z"ahlt.

\subsubsection{AP-Regenerierung}

AP werden auch durch Schlafphasen, die \textbf{k"urzer} sind
\textbf{als 4\,h}, regeneriert [DFR\,103]. Die regenerierten AP werden
anhand der bisherigen Regeln \textbf{proportional umgerechnet} auf die
reduzierte Schlafzeit (1\,h Schlaf bringt also $\frac{\sf 1}{\sf 8}$
der bei Schlafbeginn fehlenden AP zur"uck).

Es kann passieren, da"s eine Figur v"ollig \textbf{ausgeschlafen} ist,
aber dennoch viele AP verloren hat. In einem solchen Fall kann der
SpL gestatten, da"s auch durch echtes \textbf{Ausruhen} in
tiefer Entspannung und völliger Ruhe AP in reduzierter Geschwindigkeit
regeneriert werden (und zwar mehr als die 2\,AP, die in [DFR\,103]
zugestanden werden).

\subsubsection{Kritischer Schaden}

Klarstellungen zur Tabelle 4.5 \emph{Kritischer Schaden} [DFR\,246f]:

\begin{description}
\item[56--64:] \emph{Verletzung am rechten Bein.} Die getroffene Figur
  \textbf{stürzt} mangels Abstützmöglichkeit \textbf{zu Boden} (und
  verliert damit ihre ggf. noch verbleibende Handlung [DFR\,88]). Sie
  kann mit \textbf{B4} kriechen, aber ohne Abstützmöglichkeit nicht
  aufrecht stehen oder gehen. Sie kann wie ein \textbf{Liegender}
  kämpfen, ist aber natürlich weiterhin \textbf{wehrlos}.
\item[94--96:] \emph{Halstreffer.} Bei einer Verletzung der
  Halsschlagader verliert die getroffene Figur zudem das Bewußtsein,
  bis die Blutung gestillt wurde.
\end{description}

\subsubsection{Behandlung ernster Verletzungen}

Bei der magischen Behandlung von \textbf{schweren Verletzungen} oder
den Folgen von \textbf{kritischen Treffern} reduziert jeder durch die
Behandlung geheilte LP die ausgew"urfelte Dauer der schweren
Verletzung um \textbf{einen Tag}. Dies gilt sogar für LP, die dem
Verzauberten wegen Erreichen des LP-Maximums gar nicht gutgeschrieben
werden können. Davon unbeschadet bleibt die Möglichkeit, durch
separate Anwendung von \emph{Allheilung} in einem Zauberduell die
Folgen ernster Verletzungen direkt zu beseitigen [ARK\,85], wobei
hier vor dem Zaubern entschieden werden muß, ob 2W6 LP geheilt werden
sollen (mit den oben beschriebenen Auswirkungen auf die Heildauer)
oder ob die langanhaltenden Folgen einer Verletzung direkt beseitigt
werden sollen (wobei dann keine LP geheilt werden).

Die Restgenesungsdauer ernster Verletzungen kann weiterhin duch Zauber
wie \emph{Blutmeisterschaft} und den \emph{Klän\-gen der Genesung}
verkürzt werden. Ausdrücklich ausgenommen von dieser Regelung ist
allerdings die \emph{Wundrune} (deren Vorteil darin besteht, daß sie
beliebig häufig hintereinander angewendet werden kann, was aber nicht
zur vollständigen Beseitung der Folgen oben genannter Verletzungen in
wenigen Stunden mißbraucht werden soll).

% \emph{Anmerkung: Diese Regelung gab es in ähnlicher Art bereits in den
%   Hausregeln zu M3. Allerdings hat M4 dann recht eindeutig
%   klargestellt, daß sie nicht den Absichten der M4-Autoren entspricht,
%   weshalb wir sie entfernt haben. Ironischerweise hat dann aber M5 das
%   Heilen schwerer Verletzungen erleichtert (indem dies nun mit
%   \emph{Heilen schwerer Wunden} möglich ist). Dies war Anlaß genug,
%   unsere alte Hausregel wiederzubeleben.}

\subsection{Gezielte Angriffe}
\label{gezielt}

\subsubsection{Nahkampf}

W"ahrend bei echten kritischen Treffern \textbf{kein LP-Verlust}
einzutreten braucht (allerdings mu"s das Opfer \textbf{AP verlieren})
[DFR\,243], um die in Tabelle~4.5 beschriebenen Auswirkungen
hervorzurufen, gilt diese Regel bei gezielten Hieben
\textbf{nicht}. Ein gezielter Hieb erzielt nur dann den gew"unschten
kritischen Schaden, wenn das Opfer \textbf{mindestens 1\,LP} verliert
(ähnlich wie beim Meucheln) oder wenn der \textbf{EW:Angriff} einen
\textbf{kritischen Erfolg} ergeben hat und der zweite WW:Abwehr
mi"slungen ist. \textbf{Gelingt} allerdings der \textbf{zweite
  WW:Abwehr}, so ist der angerichtete Schaden auf jeden Fall nur
\textbf{leicht}, selbst wenn der EW:Angriff einen kritischen Erfolg
ergeben hat.

\emph{Anmerkung: Gemäß Standardregeln sind gezielte Hiebe zu
  m"achtig. Im Gegensatz zu den Empfehlungen des Regeltextes gibt es
  nur wenige Situationen, in denen es nicht sinnvoll ist,
  ausschlie"slich mit gezielten Hieben zu k"ampfen. Mit den hier
  vorgenommenen Regl"anderungen wird dieser Mi"sstand dadurch behoben,
  da"s ein K"ampfer, der einen gezielten Hieb landen will, auf die
  Chance verzichtet, durch diesen Hieb an anderer Stelle kritisch zu
  treffen. Durch die Bedingung, da"s kritischer Schaden nur bei
  LP-Verlust eintritt, werden die verheerenden M"oglichkeiten eines
  gezielten Hiebs weiter eingeschr"ankt. Ein "`echter"' kritischer
  Treffer simuliert gewisserma"sen das Zusammentreffen
  (un)gl"ucklicher Umst"ande, was nicht unbedingt gezielt
  herbeizuf"uhren ist.}

%   (Um ein grobes Bild
%   zu geben: Bewertet man einen schweren Treffer dreimal so gut wie
%   einen leichten Treffer und einen kritischen Treffer wiederum dreimal so
%   gut wie einen schweren Treffer, und geht man von gezielten Hieben
%   aufs Bein aus (was völlig ausreicht, um einen Gegner kampfunfähig zu
%   machen), dann ergeben sich folgende Erwartungswerte:
%   Angriff +7, Abwehr +11, gezielter Hieb 15\,\%\ besser --
%   Angriff +18, Abwehr +16, gezielter Hieb 42\,\%\ besser --
%   Angriff +18, Abwehr +11, gezielter Hieb 90\,\%\ besser --
%   Angriff +7, Abwehr +16, gezielter Hieb 10\,\%\ schlechter.)
  
Um sich bei einem gezielten Hieb als Ziel in der Tabelle auf
[DFR\,248] "`100: T"odlicher Treffer"' auszusuchen, muß es zunächst
einmal überhaupt die Möglichkeit geben, den Gegner mit der benutzten
Waffe gezielt augenblicklich zu töten. Desweiteren muß der Angreifer
von dieser Möglichkeit detailliert Kenntnis haben. Entsprechend der
gewählten Möglichkeit wird der passende Rüstungsschutz
berücksichtigt. Mit einem Dolch kann man einen Menschen durch einen
Stich ins Herz augenblicklich töten, muß dazu aber den Rüstungsschutz
des Rumpfes überwinden. Wenn der Hals weniger gut gerüstet ist, kann
man dem Opfer zwar die Halsschlagader aufschneiden. Dies zählt aber
"`nur"' als "`94--96: Halstreffer"', was nicht augenblicklich zum Tode
führt (und auch nur bei einer \emph{besonders schweren Verletzung} die
angestrebte Wirkung hat). Mit Schwert oder Schlachtbeil kann ein
Treffer am Hals sehr wohl den unmittelbaren Tod bewirken, z.\,B. durch
Genickbruch oder Köpfen, so daß hier die Wahl von "`100: T"odlicher
Treffer"' zulässig ist. Besonders interessant wird es bei Gegnern,
deren Organismus in keinster Weise mehr als menschenähnlich bezeichnet
werden kann (Drachen, Schleimmonster, \dots) oder die noch nicht eimal
mehr Lebewesen sind (Untote, Automaten, Geister, \dots). Im [BES]
finden sich nur gelegentlich Ausführungen, die hier hilfreich sind
(z.\,B. über Drachen [BES\,58ff]). Meist muß der SpL improvisieren.

Ein gezielter Hieb kann auch benutzt werden, um einen Gegner gezielt
im offenen Kampf zu \textbf{bet"auben}, ohne ihm ernsthaften Schaden
zuzuf"ugen. (Nach den normalen Bet"auben-Regeln [DFR\.160ff] ist das
nicht m"oglich, da der Gegner hierf"ur ahnunglos sein mu"s.) Dazu ist
ein \textbf{gezielter Hieb auf den Kopf} n"otig (74--80 auf der
Tabelle~4.5). Erreicht der Hieb sein Ziel, w"urfelt der Angreifer einen
\textbf{EW:Bet"auben ohne WM--Grad}. \textbf{Mi"slingt} der EW, tritt
normaler Schaden am Kopf ein, der bei Verlust von mindestens 1\,LP
kritisch ist wie unter 74--80 in der Tabelle
beschrieben. \textbf{Gelingt} der Wurf, tritt nur \textbf{leichter}
Schaden ein, und das Opfer wird f"ur \textbf{2W6 Runden bewu"stlos}
(1W6 Runden, wenn es einen Metallhelm tr"agt). Diese Art des
Bet"aubens ist wie Meucheln nur gegen hinreichend menschen"ahnliche
Gegner m"oglich. (Davon unbeschadet bleibt die M"oglichkeit des
\emph{Faustkampfes} [DFR\,197f].)

Gezielte Hiebe mit \textbf{kanthanischen Schwertern} haben nicht die
dramatischen Auswirkungen, die in [KTP\,248] beschrieben sind. Selbst
mit den Standardregeln reicht ein erfolgreicher gezielter Angriff auf
ein Bein oder den Waffenarm i.\,d.\,R. aus, um den Kampf zu beenden,
so daß hier kein spieltechnischer Bedarf besteht, die Regeln
gefährlicher zu gestalten. Die Lebensgefahr durch Verbluten wird von
den Standardregeln noch nicht einmal simuliert, wenn ein Körperteil
komplett \textbf{abgetrennt} wird. Es wäre daher unverhältnismäßig,
einer Schnittwunde ein größeres Gefährdungspotential zu geben.

\subsubsection{Fernkampf}

Die Fertigkeit \emph{Werfen} [DFR\,208] kann \textbf{nicht für gezielte
  Würfe} benutzt werden.

Da gezielte Schüsse und Würfe immer auf wehrlose Figuren erfolgen
[DFR\,249], erlangt man dabei niemals KEP. Allerdings können AEP
für das "`kampflose"' Überwinden von Gegnern vergeben werden.

Auch im Fernkampf gilt, da"s mindestens 1\,LP Schaden angerichtet
werden mu"s, um beim Opfer kritischen Schaden hervorzurufen. Dies gilt
sogar bei einem kritischen Erfolg, der die in [DFR\,250] beschriebenen
Auswirkungen hat, aber nicht erm"oglicht, kritischen Schaden ohne
LP-Verlust anzurichten.

Auch im Fernkampf gilt, da"s ein gezielter Schu"s bzw. Wurf nur dann
augenblicklich t"odlich sein kann, wenn prinzipiell die Möglichkeit
dazu besteht, der Schütze von dieser Möglichkeit weiß \emph{und}
mindestens 1\,LP Schaden angerichtet wird. Zusätzlich muß bei beiden
EW mindestens eine 30 erreicht werden. Ein Ergebnis zwischen 25 und 29
richtet "`nur"' passenden kritischen Schaden an, wie das im Beispiel
[DFR\,250] auch schon angedeutet ist. Zur Inspiration: Ein tödlicher
Schuß oder Messerwurf wird zu "`21--35: Rumpftreffer mit Gefahr
innerer Verletzungen"'. Ein tödlicher Schuß in den Hals wird zu
"`94--96: Halstreffer"'. Mit einer Schleuder, Wurfkeule oder ähnlich
stumpfen Fernwaffen ist ein augenblicklicher Tod eher nicht gezielt zu
erreichen. Bei einer erreichten 30 wäre aber "`97: Schwere
Schädelverletzung"' eine passende Auswirkung, die bei 25--29 zu
"`74--80: Kopftreffer"' abgemildert wird.

Der in [BES\,60] beschriebene gezielte Fernkampfangriff auf die
verwundbare Stelle eines Drachens ist entsprechend auch nur möglich,
wenn der Schütze sehr genau von der Existenz und dem Ort dieser Stelle
weiß. Für einen direkt tödlichen Schuß ist noch genauere Kenntnis
nötig.

Analog zum Kombinieren von speziellen Kampftechniken im Nahkampf
(siehe Abschnitt \ref{kombi}) können schnelle Fernkampfwaffen, die
mehrere Angriffe pro Runde erlauben (z.\,B. Wurfsterne), nur für einen
Angriff pro Runde benutzt werden, wenn mit ihnen ein gezielter Schuß
oder Wurf durchgeführt wird.

\subsection{Spezielle Kampftechniken}

\subsubsection{Sturmangriff}
\label{sturm}

Klarstellend zu [DFR\,230] ist ein Angriff genau dann ein
Sturmangriff, wenn beide folgende Punkte gegeben sind:
\begin{itemize}
\item Es handelt sich um einen Angriff, der im Zuge der Aktionen 2d,
  2f, 2g oder 2h [DFR\,220] durchgeführt wird, aber \textbf{nicht} um
  einen Zusatzangriff (genau definiert durch Punkt \ref{i:zusatz} in
  Abschnitt \ref{kombi}).
\item Der Angreifer hat sich um mindestens 6m und höchstens die Hälfte
  seiner Bewegungsweite bewegt \textbf{oder} der Angriff erfolgt gegen
  eine Figur, die ihrerseits einen Sturmangriff gegen den Angreifer
  durchführt.
\end{itemize}

\subsubsection{Konzentrierte Abwehr}

Ein \emph{Wehrloser} kann niemals konzentriert abwehren (auch nicht
mit einem großen Schild).

Wird eine konzentrierte Abwehr durchgeführt, betrifft die negative WM
\textbf{alle eigenen EW:Angriff und äquivalente EW} der laufenden
Runde mit Ausnahme der in Abschnitt \ref{kombi} in Gruppe
\ref{i:zusatz} genannten Zusatzangriffe. (Für \emph{Fechten} ist
das ausdrücklich in [DFR\,140] geregelt.)

\subsubsection{Zurückdrängen}

Um einen Gegner \textbf{zurückzudrängen} [DFR\,228] muß der eigene
gelungene Nahkampfangriff dem Gegner mindestens 1\,AP geraubt haben.

\subsubsection{Peitsche}

Der durch den Peitschenhieb ausgelöste Schock [DFR\,204] löst Stufe 2a
auf der Tabelle zur Handlungsunfähigkeit am Ende dieser Hausregel aus,
mit der zusätzlichen Regel, daß das Opfer sich in der nächsten Runde
nicht bewegen kann. Man beachte die Ausführungen zum genauen
Zeitablauf im Abschnitt~\ref{timing}.

\subsubsection{Kampfgabeln}

\textbf{Waffenrang} und \textbf{Rüstungsbonus} von Kampfgabeln sind
die gleichen wie bei einem \textbf{Dolch}.

Die in [KTP\,246] erwähnte Eigenschaft des \textbf{Sai}, einzeln auch
als \textbf{Parierdolch} eingesetzt werden zu können, ist nicht
spezifisch für das Sai, sondern gilt für alle Formen der Kampfgabel,
selbst dann, wenn in der anderen Hand eine andere Waffe als eine
Kampfgabel geführt wird (vgl. [DFR\,214]). Allerdings wird immer der
Erfolgswert für \textbf{Kampfgabeln} zugrundegelegt, nicht der für
\emph{Parierdolch}.

\subsubsection{Waffenloser Kampf}
\label{wlk}

Der Zuschlag zum WW:Abwehr gegen Nahkampfangriffe beim
\emph{waffenlosen Kampf} wird nur in Runden angerechnet, in denen der
Kämpfer diese Waffenfertigkeit auch wirklich anwendet. Letzteres muß
sich nicht unbedingt durch einen tatsächlichen EW:Angriff mit
\emph{waffenlosem Kampf} definieren und kann vom SpL durchaus flexibel
gehandhabt werden. Der Zuschlag kommt aber insbesondere dann nicht zur
Geltung, wenn der Kämpfer noch eine Waffe in der Hand hält
(ausgenommen Tonfa, Kampfriemen und YoTenTori), sich in der laufenden
Runde um mehr als die Hälfte seiner Bewegungsweite bewegt hat,
\emph{Faustkampf} betreibt oder sich einer gänzlich anderen Tätigkeit
widmet.

Entsprechendes gilt für den zusätzlichen WW gegen das Einleiten eines
Handgemenges. Dieser kann außer durch \emph{waffenlosen Kampf} auch
durch \emph{Faustkampf} herbeigeführt werden. Allerdings muß der
Kämpfer sich für eine der beiden Möglichkeiten entscheiden.

\subsubsection{Faustkampf}

Siehe Anmerkungen in Abschnitt \ref{wlk} zu den Bedingungen, unter
denen ein zusätzlicher WW:\emph{Faustkampf} gegen das Einleiten eines
Handgemenges gewährt wird.

Der zweite WW:Abwehr des Opfers eines \emph{Faustkampf}-Angriffs
erhält genau die gleichen WM wie der erste WW:Abwehr
\textbf{zuzüglich} der negativen WM durch den AP-Verlust. Wenn dem
Opfer gar kein WW:Abwehr zustand, erfolgt auch kein zweiter
WW:Abwehr. Gelingt der zweite WW:Abwehr, ist das Opfer in der
folgenden Runde auch nicht zu Aktionen fähig, die ähnliche oder gar
höhere Anforderungen an die Konzentrationsfähigkeit stellen wie ein
Angriff, insbesondere \emph{Zaubern}. Diese negativen Effekte setzen
auch ein, wenn der zweite WW:Abwehr mißlingt, und zwar in diesem Falle
sofort, nicht erst in der folgenden Runde (laufende Zaubervorgänge
werden abgebrochen). Für den genauen Dauer der resultierenden
Wehrlosigkeit und Bewußtlosigkeit vgl. Abschnitt~\ref{timing}.

\subsubsection{Kampfriemen}

Die (westlichen) Kampfriemen können nicht in Verbindung mit
\textbf{KiDo} benutzt werden. (Stattdessen werden \textbf{YoTenTori}
benutzt, siehe Abschnitt \ref{yotentori}.) Die Behinderungen durch
angelegte Kampfriemen entsprechen denen durch YoTenTori [KTP\,251].

\subsubsection{Tonfa}

Ein Kampfstab kann auch zur Abwehr von Tonfa-Angriffen eingesetzt
werden.

Die WM+1 auf EW:Angriff durch \emph{TaiTschi} gilt auch für das Tonfa.

Das Tonfa gibt zwei unabhängige Zuschläge auf den WW:Abwehr: Zum einen
bei hohem Erfolgswert gegen Nahkampfangriffe aus allen Richtungen
(analog zum \emph{waffenlosen Kampf}, zum anderen den Erfolgswert für
Tonfa abzüglich 5 gegen Angriffe von vorne und von der Seite des
abwehrenden Tonfa (analog zum Kampfstab). Der letztere Zuschlag wird
nur gewährt gegen Angriffe, gegen die auch ein Kampfstab verwendet
werden kann. Beide Zuschläge addieren sich.

Waffenrang und Rüstungsbonus sind die gleichen wie beim Kampf mit
bloßen Händen. Ebenso sind (ohne entsprechende KiDo-Techniken) keine
gezielten Hiebe möglich.

Die Anmerkungen in Abschnitt \ref{wlk} gelten sinngemäß auch für das
Tonfa.

\subsubsection{YoTenTori}
\label{yotentori}

Die WM+1 auf EW:Angriff durch \emph{TaiTschi} gilt auch für das YoTenTori.

Analog zum Kampf mit \textbf{Kampfriemen} haben YoTenTori im
Handgemenge keinen Effekt, d.\,h. die Figur kämpft im Handgemenge mit
dem Erfolgs- und Schadenswert von \emph{waffenlosem Kampf}.

Waffenrang und Rüstungsbonus sind die gleichen wie beim Kampf mit
bloßen Händen. Ebenso sind (ohne entsprechende KiDo-Techniken) keine
gezielten Hiebe möglich.

Die Anmerkungen in Abschnitt \ref{wlk} gelten sinngemäß auch für die
YoTenTori.

\subsubsection{Kombination spezieller Angriffsweisen}
\label{kombi}

\fbox{\parbox{0.98\textwidth}{%
  \textbf{Beispiel:} \emph{Beowulf beschleunigt sich magisch
    und stürmt dann seinen Gegnern entgegen. Er verbraucht dabei knapp
    90\,\%\ seiner Bewegungsweite, aber dies läßt ihm noch Gelegenheit
    für einen überstürzten Hieb. Diesen führt er als Rundumschlag gegen
    drei Gegner. Er schlägt dabei gezielt auf den Hals jedes
    einzelnen, um alle drei auf einen Streich zu köpfen. Da er
    beschleunigt ist, führt er gleich zwei dieser überstürzten
    gezielten Rundumschläge hintereinander aus, auf insgesamt
    sechs verschiedene Gegner, die er mit einer großen Portion Glück
    dann auch alle ins Jenseits befördern könnte. Und ach ja, diese
    Angriffe werden alle noch ein wenig erschwert, um konzentriert
    abwehren zu können\dots}
}}

Natürlich funktioniert das oben beschriebene Manöver so nicht. Leider
wird das aus dem Regelwerk nicht gerede deutlich. Es gilt das
Grundprinzip, daß sich verschiedene Spezialregeln für einen Angriff
nicht miteinander kombinieren lassen. Ein Rundumschlag kann weder
gezielt noch überstürzt sein, und ein überstürzter Hieb kann
selbstverständlich auch nicht gezielt ausgeführt werden. Es gibt
allerdings auch Ausnahmen von diesem Grundprinzip. Die folgenden
Richtlinien versuchen, die Handhabung etwas klarer zu gestalten.

Die speziellen Angriffsweisen lassen sich in Gruppen einteilen:

\begin{enumerate}
\item \label{i:zusatz} Zusatzangriffe, die ohne die jeweilige
  Spezialregel gar nicht möglich wären: Spontaner Hieb bzw. Angriff,
  überstürzter Angriff, zusätzlicher Angriff (bei 61--70 als Folge
  eines kritischen Fehlers des Gegners bei dessen Abwehr oder
  Angriff), zusätzlicher Angriff mit einer Parierwaffe, zusätzlicher
  Angriff durch \emph{Fechten}, spontaner Gegenangriff mit
  \emph{SoJutso}. Hier gilt streng, daß diese Angriffe mit keinerlei
  anderen Spezialregeln kombiniert werden können. Allerdings wird ein
  ggf. in der gleichen Runde durchgeführter normaler Angriff davon
  nicht berührt. Solch ein Angriff kann weiterhin durch die speziellen
  Angriffsweisen aus Gruppe \ref{i:ersatz} ersetzt
  werden. Insbesondere gilt dies für die beiden zuletzt genannten
  Möglichkeiten, die ja erst möglich sind, wenn zunächst ein regulärer
  Angriff (oder eine konzentrierte Abwehr) durchgeführt wird
  (z.\,B. könnte nach einem gezielten Rapierangriff ein Zusatzangriff
  mit \emph{Fechten} durchgeführt werden).
\item \label{i:ersatz} Spezielle Angriffe, für die die Aktion 2d
  [DFR\,220] (\emph{mit einer Nahkampfwaffe angreifen}) gewählt werden
  muß, wobei dann aber der übliche normale Angriff durch den
  speziellen Angriff ersetzt wird: gezielter Angriff (inklusive der
  Anwendung von \emph{BaiTeng}), Rundumschlag, Entwaffnen (nur mit
  bestimmten Waffen möglich), Mehrfachangriff mit schneller Waffe
  (z.\,B. NunChaku), Anwendung von \emph{Kampf in Dunkelheit},
  \emph{beidhändigem Kampf}, \emph{Kampf in Schlachtreihe}, Anwendung
  einer \emph{KiDo}-Angriffstechnik oder \emph{NiTo}. Angriffe, die
  nicht im Zuge von Aktion 2d durchgeführt werden (also z.\,B. aus
  Gruppe \ref{i:zusatz} oder \ref{i:aktion}), können daher nicht mit
  diesen Angriffsweisen kombiniert werden. Weniger offensichtlich ist,
  daß die Angriffsweisen aus dieser Gruppe auch \textbf{nicht
    miteinander} kombiniert werden können. Beispiele: Ein Rundumschlag
  kann nicht gezielt ausgeführt werden. Wenn ein beidhändiger Kämpfer
  im Dunkeln kämpft, muß er sich entscheiden, ob er einhändig
  \emph{Kampf in Dunkelheit} anwendet oder lieber mit
  \emph{beidhändigem Kampf} angreift, dafür aber die übliche WM-6 in
  Kauf nimmt.
\item \label{i:aktion} Angriffe, die eine spezielle Aktionswahl nach
  [DFR\,220f] erfordern: \emph{Konzentrierte Abwehr} (2f), \emph{den
    Gegner zu Fall bringen} (2g), \emph{ein Handgemenge einleiten}
  (2h). Diese Aktionen sind einfach keine normalen Angriffe und können
  daher nicht durch Angriffsweisen aus Gruppe \ref{i:ersatz} ersetzt
  werden. Zusatzangriffe aus Gruppe \ref{i:zusatz} sind aber weiterhin
  möglich, wenn das ihre jeweilige Regel gestattet. Bei einer
  konzentrierten Abwehr gilt z.\,B. ausdrücklich, daß durch Einsatz
  von \emph{Fechten} ein Zusatzangriff durchgeführt werden kann
  [DFR\,140]. Analog ist ein Zusatzangriff mit einer Parierwaffe
  gestattet.
\end{enumerate}

Zu beachten ist, daß ein \textbf{Sturmangriff} keine spezielle
Angriffsweise im Sinne dieses Abschnitts ist. Gemäß Abschnitt
\ref{sturm} zählen automatisch Angriffe aus Gruppe \ref{i:ersatz} und
\ref{i:aktion} als Sturmangriff, sobald die Voraussetzungen dafür
erfüllt sind. Ähnliches gilt für Angriffe mit improvisierten und/oder
unvertrauten Waffen. Hier gibt es keinerlei Einschränkungen bei der
Anwendung spezieller Angriffstechniken aus den oben genannten Gruppen.

Der Zauber \emph{Beschleunigen} verdoppelt prinzipiell die
Handlungsmöglichkeiten, unbeschadet eventuell gewählter spezieller
Angriffsmöglichkeiten. Es wäre also in der Tat möglich, beschleunigt
z.\,B. zwei Rundumschläge in einer Runde auszuführen (der Einfachheit
halber beide am Ende der Runde durchzuführen, solange dies nicht zu
unrealistischen Abläufen führt). Zusatzangriffe aus Gruppe
\ref{i:zusatz} ergeben sich eher aus der Gelegenheit als aus dem Tempo
und werden daher nicht verdoppelt, wobei \emph{Fechten} und
Zusatzangriffe mit der Parierwaffe hier eine Sonderrolle spielen und
durchaus verdoppelt werden können. Entsprechendes gilt für den Zauber
\emph{Verlangsamen}.

Der bei Anwendung von \emph{IaiJutsu} [KTP\,161f] in der gleichen
Runde wie das Ziehen der Waffe erfolgende Angriff erfolgt zwingend
durch Aktion 2d, kann dann aber auch durch spezielle Angriffe aus
Gruppe \ref{i:ersatz} ersetzt werden.

Die Parade beim \emph{SoJutsu} [KTP\,172] wird genauso behandelt wie beim
\textbf{Fechten}, d.\,h. der in der gleichen Runde erfolgende
stechende Angriff wird normal als Aktion 2d oder 2f abgehandelt und
kann im ersteren Fall durch spezielle Angriffe aus Gruppe
\ref{i:ersatz} ersetzt werden. Gleiches gilt für den speziellen
\emph{SoJutsu}-Angriff mit dem Yari. (\emph{NiTo} [KTP\,168f] wird
hier ausdrücklich nicht analog zum \emph{Fechten} behandelt, weil es
den Angriff selbst modifiziert und daher als Spezialangriff der Gruppe
\ref{i:ersatz} betrachtet wird.)

Bei den Handlungen im \emph{Handgemenge} [DFR\,239] ist zu beachten,
daß es sich hier um grundsätzlich andere Aktionen handelt als im
normalen Nahkampf. Somit kommen die speziellen Aktionen aus Gruppe
\ref{i:aktion} gar nicht erst in Frage. Auch die Angriffsweisen aus
Gruppe \ref{i:ersatz} sind nicht möglich, weil ja kein normaler
Nahkampfangriff durchgeführt wird, denn man ersetzen
könnte. Zusatzangriffe aus Gruppe \ref{i:zusatz} ergeben sich nur
durch Folgen kritischer Fehler.

Ein Angriff von außen in ein Handgemenge hinein ist zwar an sich ein
normaler Angriff, dennoch ist dabei aber aus Plausibilitätsgründen
kein Rundumschlag möglich. Ein gezielter Hieb ist möglich, wobei immer
dann, wenn nicht das beabsichtigte Ziel getroffen wird (nach einem
Mißerfolg beim EW:Angriff bzw. zufällig beim Angriff mit zweihändigen
Hiebwaffen) der Angriff normal (also nicht gezielt) abgehandelt wird.

Beim \emph{Meucheln} wird zwar ein EW:Angriff durchgeführt, dennoch
zählt dies aber als eine spezielle Aktion, die nicht mit einem
normalen Nahkampfangriff gleichgesetzt werden kann und bei der
grundsätzlich keine Kombinationen mit irgendwelchen speziellen
Kampfregeln möglich sind. Eine Ausnahme kann allenfalls bei
\emph{Kampf in Dunkelheit} gemacht werden.

\emph{KiDo} ist ein Spezialfall und wird im folgenden Abschnitt
\ref{kido} behandelt.

\subsubsection{KiDo}
\label{kido}

Wesen die nur durch magische Waffen verletzt werden können, können
auch durch KiDo-Fertigkeiten verletzt werden.

Man beachte, daß morgendliches (und wohl auch erfolgreiches)
\emph{TaiTschi} die Voraussetzung für die Anwendung von KiDo-Techniken
ist [KTP\,174]. Der SpL kann dies aber großzügig auslegen, was genauen
Zeitpunkt (wann genau ist "`morgens"'?) und die „Wirkungsdauer“
(durchaus etwas länger als nur bis zum nächsten Morgen) des
Schattenboxens angeht.

Die Sonderregel zum Verlust aller AP durch die Anwendung von
\emph{KiDo} ist so zu interpretieren, daß die SpF vor dem EW:KiDo noch
mehr als 0~AP \textbf{und} genügend AP für die anzuwendende
KiDo-Technik haben muß und dann \textbf{durch die Anwendung} auf 0~AP
reduziert wird. Nur dann werden die Nachteile bis zum Ende der
laufenden Runde ignoriert.

Für die Kombination von KiDo mit anderen speziellen Kampftechniken
(vgl. Abschnitt \ref{kombi}) gilt der Grundsatz, daß nur die
Kombinationen möglich sind, bei denen die Regeln das explizit vorsehen
(vgl. [KTP\,181] und die Hinweise, daß \emph{NiTo}
"`selbstverständlich"' und \emph{SoJutsu} "`natürlich"' nicht mit
KiDo-Techniken kombiniert werden können). Die Anwendung von
KiDo-Techniken vom Pferderücken oder vom Streitwagen ist grundsätzlich
ausgeschlossen.

Bei KiDo-Techniken aus der Gruppe \emph{Werfen} erhält der Gegner
WM--4 auf seinen WW:Abwehr, wenn er \textbf{in derselben Runde} einen
aktiven Sturmangriff oder eine KiDo-Technik aus der Gruppe
\emph{Anspringen} gegen den Kämpfer durchgeführt hat oder
durchzuführen im Begriff ist.

Techniken, die den Einsatz von \emph{waffenlosem Kampf} verlangen,
können grundsätzlich auch mit \emph{Tonfa} oder \emph{YoTenTori}
angewendet werden. Für den Angriff wird dann der Erfolgs- und
Schadenswert der entsprechenden Waffe benutzt. Die diesbezüglichen
Angaben in den Beschreibungen der jeweiligen Techniken sind recht
inkonsistent und werden durch diese Regelung ersetzt, allerdings mit
folgenden Ausnahmen:
\begin{itemize}
\item \emph{DoMino}, \emph{HoiChian}, \emph{KaguTobu} und
  \emph{KaruRikiHa} können nicht mit Tonfa oder YoTenTori angewendet
  werden, da für diese Techniken beide Hände völlig frei sein müssen.
\item \emph{KamaKusa} schließt YoTenTori ausdrücklich aus.
\item Bei \emph{TsuraSone} kann die Waffe mit YoTenTori nicht
  übernommen werden. Tonfas können dafür ggf. fallengelassen werden.
\end{itemize}

\section{Kreaturen}

Die Regeln zu der reduzierten AP-Zahl von bestimmten kleinen
menschenähnlichen Kreaturen werden abgewandelt. Solche kleinen
Kreaturen besitzen nicht ein bestimmtes Vielfaches ihres Grades
weniger AP, sondern einen bestimmten Bruchteil der AP, die ein Mensch
auf gleichem Grad und gleicher Charakterklasse h"atte. Dieser
Bruchteil wird vor Addieren des AP-Bonus ber"ucksichtigt. Dabei wird
stets abgerundet. F"ur die verschiedenen Spezies ist er hier
aufgelistet:
\begin{description}
\item[Dunkelzwerg:] 2/3
\item[Fee:] 1/3
\item[Gnom:] 1/2
\item[Halbling:] 2/3
\item[Kobold:] 1/2
\item[Vogelmensch:] 5/6
\end{description}

Bei den Entbehrungsregeln wird der Faktor entsprechend auf die
gradabhängigen Verluste angewandt.

Der Einsatz von \textbf{angeborenen Fähigkeiten, die wie Zauberei
  wirken} [BES\,20], zählt grundsätzlich als Handlung, die das Wesen
normalerweise eine ganze Runde beschäftigt, auch wenn keine
Konzentration nötig ist (und damit weiterhin Angriffe abgewehrt werden
können, was aber keine Handlung darstellt). Wie bei M5 klargestellt,
ist die Zauberdauer grundsätzlich 1\,sec (so nicht ausdrücklich anders
geregelt).

Zu gezielten Angriffen auf \textbf{Drachen} vergleiche die
Ausf"uhrungen im vorigen Abschnitt. Ein Bad im \textbf{Drachenblut}
unter Wirkung des Zaubers \textbf{Hitzeschutz} richtet \textbf{3W6}
schweren Schaden an.

Ein \textbf{Hyrrbauti} [MDS\,254] explodiert mit "`Zaubern+18"'.

Das Umhüllen durch \textbf{Landrochen} [BES\,180ff] wird die das
Umfließen beim \textbf{Riesenschleim} [BES\,277] gehandhabt,
d.\,h. das Opfer kann als einzige Handlung \emph{sich aus dem
  Handgemenge lösen}.

Eine \textbf{Weberin} [MDS\,280] kann von einem Seelenreisenden
ausnahmsweise mit \emph{waffenlosem Kampf} bekämpft (aber nicht
in ein Handgemenge verwickelt) werden. Eine Halsfesselung führt bei
Seelenreisenden nicht zum Ersticken.

\section{Magie}

\subsection{Zaubern}

Die WM-2 auf Zauber, die weder Grundzauber noch Spezialgebiet eines Magiers
sind, gelten für sämtliche Zauberkünste, nicht nur für Sprüche im eigentlichen
Sinne. Dabei ist zu beachten, daß Zauberlieder für Barden als Grundzauber
zählen ebenso wie sämtliche Zaubersiegel, Runenstäbe, Schutzrunen, Große
Siegel und Zauberlettern für Thaumaturgen.

Gemäß [ARK\,21] ist ein Zauberer auch bei einem 1-sec-Zauber die ganze Runde
über wehrlos (und kann sich auch nur 1\,m weit bewegen [DFR\,88]). Allerdings
stört auch ein erfolgreicher Angriff in der Runde nicht die Ausführung des
1-sec-Zaubers, weil die eigentliche Konzentration auf den Zauber selbst nur
einen Augenblick dauert und somit nicht gestört werden kann. Natürlich gilt
auch weiterhin, daß der Zauberer zum Zeitpunkt des Zauberns noch über
ausreichend viele AP zum Zaubern und ausreichend viele LP zum
Handlungsfähigsein (i.\,d.\,R. mehr als 3\,LP) verfügen muß.

Klarstellung zu [ARK\,25]: Gegen einen kritischen Erfolg beim EW:Zaubern
hilft weiterhin auch nur ein kritischer Erfolg beim WW:Resistenz. Wenn
ein kritischer Erfolg beim EW:Zaubern mit einem kritischen Fehler beim
WW:Resistenz zusammentrifft, bleibt es bei der Verdopplung, die aus
dem kritischen Erfolg beim EW:Zaubern resultiert. Es wird keine
weitere Verdopplung für den kritischen Fehler beim WW:Resistenz
vorgenommen.

\subsection{Konzentration}

Eine Vielzahl an Zaubern erfordert, daß der Zauberer (oder ggf. der
Benutzer) sich auch nach Abschluß des Rituals zur Aufrechterhaltung
des Zaubers während der wirkungsdauer \textbf{konzentrieren} muß. Es
ist unklar, ob damit die gleiche Art den Konzentration gemeint ist wie
\textbf{während} des Zauberns, die zu Wehrlosigkeit und B1 führt
[ARK\,21]. Einige Sprüche spezifizieren die Situation für den
jeweiligen Zauber genauer, siehe z.B. \emph{Unsichtbarkeit}
[ARK\,174]. Man könnte nun annehmen, daß für alle nicht spezifizierten
Aspekte das gleiche gilt wie während des Zauberrituals. Die
Beschreibung von \emph{Wirbelwind} [ARK\,190] legt z.\,B. nahe, daß B1
nicht nur bei diesem Spruch die Bewegungsweite beim Konzentrieren
ist. Allerdings sagt \emph{Unsichtbarkeit} nichts über die
Bewegungsweite aus, und Unsichtbare auf B1 zu beschränken erscheint
nicht sinnvoll. M5 regelt die Angelegenheit sehr genau und konsistent,
allerdings ist eine Rückübertragung nach M4 nicht ohne größere
Komplikationen möglich. Die folgende Regelungen für die Konzentration
zur Aufrechterhaltung eines Zaubers nach Abschluß des Rituals dienen
der Spielbarkeit, entsprechen weitestgehend dem Geist und dem Wortlaut
von M4 und sind dennoch ein wenig von M5 inspiriert:

\begin{itemize}
\item Spezielle Angaben in der Zauberbeschreibung gelten
  weiterhin und mit höherer Priorität als die folgenden Punkte.
\item Die Konzentration wird auf die gleichen Arten und Weisen
  \textbf{gebrochen} wie während des Zauberns [ARK\,20]. Ggf. liegt
  die Entscheidung im Ermessen des SpL.
\item Während des Konzentrierens gilt \textbf{B1}, und es können keine
  Handlungen ausgeführt werden, deren Gelingen mit einem \textbf{EW}
  entschieden wird. \textbf{WW} inkl. \textbf{Abwehr} sind allerdings
  möglich.
\item Zusätzlich gilt, daß bei \emph{Unsichtbarkeit} die volle
  Bewegungsweite genutzt werden kann.
\item Geht es um die Konzentration auf Wesen oder Objekte (prominentes
  Beispiel \emph{Feuerkugel}) ist auch ein allenfalls kurzzeitig
  unterbrochener Sichtkontakt nötig (ähnlich wie beim Zaubern).
  % Aus den Regeln geht dies nicht wirklich hervor. Allerdings ist das
  % Explodieren einer nicht mehr im Blickfeld befindlichen Feuerkugel
  % in den FAQ zum Arkanum erwähnt.
\end{itemize}

Für die Erteilung von Anweisungen bei \emph{Macht über Menschen},
\emph{Macht über die belebte Natur} und \emph{Macht über magische
Wesen} gilt Ähnliches, auch wenn die Spruchbeschreibungen nicht
ausdrücklich Konzentration erwähnt. Während des Erteilens von
Anweisungen kann ähnlich wie bei \emph{Unsichtbarkeit} die volle
Bewegungsweite genutzt werden.

\subsection{Wiederholtes Verzaubern}

In [ARK\,29] wird geregelt, daß sich gleichartige Wirkungen
unterschiedlicher Zauber nicht addieren, ausdrücklich auch bei
Kombination von Zaubersprüchen, Trünken und Artefakten. Dies ist
weitgehend auszulegen. Modifikationen des gleichen Wertes sind in der
Regel ein gutes Kriterum für gleichartige Wirkungen. Allerdings gilt
dies nur für die direkte magische Modifikation eines
Wertes. Abgeleitete Modifikationen (wie z.\,B. erhöhter SchB durch
magisch erhöhte St) werden zusätzlich
angewendet. \textbf{Gegensätzliche} Modifikationen werden hingegen
immer gegeneinander aufgerechnet, wobei weiterhin gilt, daß das Wirken
direkt gegeneinander gerichteter Zauber durch ein \textbf{Zauberduell}
[ARK\,31] bestimmt wird.

Zur Illustration einige Beispiele:

\begin{itemize}
\item Eine Figur wurde mit \emph{Schwäche} verzaubert. Wird nun auch
  \emph{Stärke} auf die Figur gezaubert, entscheiden ein Zauberduell,
  welcher Spruch wirkt. Das gleiche gilt für \emph{Heiligen Zorn},
  wobei die gegensätzliche Wirkung der Sprüche hier zwar nicht gleich
  groß ist, aber dennoch gleichartig.
\item Eine Figur wurde gleichzeitig mit \emph{Wagemut}, \emph{Heiliger
    Zorn} und \emph{Segnen} verzaubert und verteidigt
  ihren schwer verletzten \emph{Blutsbruder} mit einem magischen
  Dolch*(+1/+1) und einem verfluchten Schild*(--1) gegen einen Gegner
  in verfluchter KR*(+1). Folgende Modifikationen ergeben sich:
  \begin{itemize}
  \item \textbf{PW:St:} Die Erhöhung durch \emph{Heiligen Zorn} wirkt bei
    PW zusätzlich zur WM--5 durch \emph{Segnen}, da dies
    unterschiedliche Arten der Modifikation sind.
  \item \textbf{Schaden:} \emph{Heiliger Zorn} erhöht den SchB
    indirekt. Daher wirkt diese Erhöhung auf den Schaden zusätzlich
    zur besten direkten magischen Erhöhung, also die +2 durch
    \emph{Blutsbrüderschaft}. Die +1-Modifikationen durch den
    Dolch*(+1/+1) und den \emph{Wagemut} haben keine Auswirkung.
  \item \textbf{EW:Angriff:} Es gilt WM+2 (durch
    \emph{Blutsbrüderschaft} bzw. \emph{Wagemut}). Die jeweiligen WM+1
    durch den Dolch*(+1/+1), durch \emph{Segnen} und die verfluchte
    Rüstung des Gegners kommen nicht zum Tragen.
  \item \textbf{WW:Abwehr:} Die höchste negative WM--2 durch
    \emph{Wagemut} wird mit der höchsten positiven WM+2 duch die
    Blutsbrüderschaft verrechnet, Ergebnis WM$\pm$0. Der verfluchte
    Schild und \emph{Segnen} haben jeweils keine Wirkung.
  \item \textbf{EW:Raufen:} Man könnte meinen, die WM+2 für
    \emph{Raufen} durch den \emph{Heiligen Zorn} wären ein indirekter
    Effekt der St-Erhöhung. Der würde allerdings anders berechnet
    werden. Es ist davon auszugehen, daß die WM+2 ein direkter
    magischer Effekt ist, die die zusätzliche Anwendung der anderen WM
    (\emph{Blutsbrüderschaft}, \emph{Segnen}) verhindert. (Ähnliches
    gilt für \emph{Stärke}, auch wenn hier die Berechnung scheinbar
    korrekt ist, nämlich WM+1 auf \emph{Raufen} durch St-Erhöhung um
    20. Allerdings würde die WM+1 auch gelten, wenn die St sich wegen
    der Begrenzung auf 100 um weniger als 20 erhöhen würde.)
  \end{itemize}
\item Der Schaden, den eine mit \emph{Wachsen} vergrößerte Figur mit
  einer magischen Waffe erzielt, wird sowohl durch den SchB+8 als auch
  durch einen eventuellen magischen SchB der Waffe beeinflußt. Aus der
  Spruchbeschreibung könnte man entnehmen, daß der SchB+8 ein direkter
  magischer Effekt ist. Allerdings handelt es sich hier um keine
  Modifikation, sondern um eine absolute Festlegung. Außerdem wird
  klar gesagt, daß der eigentliche magische Effekt das Wachsen und
  damit der St-Wert eines Riesen ist. Demzufolge würde aber
  \emph{Stärke} oder \emph{Heiliger Zorn} keine weitere St-Erhöhung
  bewirken (wohl aber die Erhöhung für \emph{Raufen}).
\item \emph{Elfenklinge} [KOM\,63] erhöht ausdrücklich den magischen
  Bonus für Angriff und Schaden um +1 und wirkt daher auch zusätzlich
  zu so einem Bonus.
\item Hingegen ist der Schadensbonus der \emph{Flammenklinge} nicht
  kumulativ mit einem eventuellen magischen Schadensbonus der
  verzauberten Waffe.
\end{itemize}

\subsection{Räumliches Bezugssystem}

Bei manchen Zaubern ist das räumliche Bezugssystem nicht leicht zu
bestimmen. Nehmen wir an, ein Zauberer ist in geeigneter Weise auf
einem schnell dahinfahrenden Wagen angegurtet und zaubert nun eine
\emph{Feuerkugel} oder gar \emph{Auflösung}. Man könnte nun die
Bewegung der enstehenden Feuerkugel (B3) bzw. Auflösungssphäre (B12)
relativ zum Erdboden verstehen, so daß der Zauberer sie mit großer
Relativgeschwindigkeit auf einen Verfolger lenken könnte. Die Bewegung
der magischen Kugeln geschieht jedoch zunächst erstmal relativ zum
Bezugssystem des Zauberers, jedenfalls solange dieser "`passiv"'
mitbewegt wird (wie im Beispiel durch den Wagen -- die Kugeln würden
also dem Wagen folgen und können relativ zum Wagen nur mit B3 bzw. B12
bewegt werden). Sollte der Zauberer sich allerdings "`aktiv"' bewegen
(was bei diesen Zaubern nur mit B1 geht, weil sonst die Konzentration
gebrochen würde und die Feuerkugel explodieren bzw. die
Auflösungssphäre implodieren würde), so bewegt er sich selbst relativ
zu dem Bezugssystem, das für die Bewegung von Feuerkugel
bzw. Auflösungssphäre relevant ist. (So ergibt sich, daß sich der
Zauberer maximal mit B4 von der Feuerkugel entfernen kann -- die B3
der Feuerkugel plus die eigene B1.)

Dieses Prinzip ist jedoch nicht auf alle Zauber uneingeschränkt
anwendbar. Eine \emph{Steinkugel} interagiert z.\,B. viel stärker mit
der physikalischen Umgebung als eine \emph{Feuerkugel}. Sie würde
sich anfänglich zwar auch mit dem Zauberer bewegen, so daß sie
zunächst auf dem Wagen ruhen und nicht von sich aus von der
Ladefläche rollen würde. Ließe sie der Zauberer aber bewußt von der
Ladefläche rollen (oder zauberte er sie von Anfang an so, daß sie
neben der Ladefläche entstünde und dann zu Boden fiele), dann würde
sie durch die Reibung mit dem Boden schnell ihre
Anfangsgeschwindigkeit verlieren. Der Zauberer könnte sie dann in der
Tat mit B12 relativ zum Erdboden einem Verfolger entgegenrollen
lassen.

% Siehe Klarstellung von Elsa: http://www.midgard-forum.de/forum/threads/6665-R%C3%A4umlicher-Bezugspunkt-f%C3%BCr-Magie

\subsection{Sekundengenaue Handlungsabläufe}

Die sekundengenaue Simulation von Handlungsabläufen [DFR\,89ff] ist
leider alles andere als vollständig geregelt. Wie schon dort
geschrieben, sollten diese Regeln wirklich nur in Ausnahmefällen
angewendet werden, wo dies nicht zu vermeiden ist, und auch dann nur
für einen Zeitraum, der so kurz wie möglich zu halten ist.

Speziell beim Einsatz von Magie gibt es einige Komplikationen, für die
die folgenden Hinweise hilfreich sein mögen:

\begin{itemize}
\item Auch bei sekundengenau simulierten Handlungsabläufen gibt es
  noch Runden mit Rundenbeginn und Ende (wobei die Runden eben
  lediglich in zehn einzelne Sekunden eingeteilt worden sind). Dies
  ist wichtig für viele die Magie betreffende Regeln.
\item Auch in den 1W6 sec vor Beginn seiner Handlung (nach gescheitertem
  PW:Gewandtheit) ist der Zauberer bereits wehrlos, wenn die geplante Handlung
  Zaubern ist. Erst wenn er den Zeitpunkt erreicht, zu dem er sich
  wieder weiter als 1m pro Runde bewegen darf, ist er nicht mehr wehrlos.
\item Störungen in den 1W6 sec vor Beginn der Handlung stören
  allerdings noch nicht die Konzentration des Zauberers. Dies ist nur
  während der eigentlichen Zauberdauer möglich.
\item Bei 1-sec-Zaubern kann die Konzentration nicht gestört werden.
\item Der Zauberer kann nach wie vor (auch während der 1W6 sec) den
  Zaubervorgang abbrechen, um eine Abwehr machen zu können (muß sich dafür
  aber wie üblich vor Würfeln des Angriffs entscheiden). Die AP für den Zauber
  verliert er auch dann, wenn der Abbruch des Zaubervorgangs schon in den 1W6
  sec erfolgt. 
\end{itemize}

\subsection{Thaumaturgie}

Der Tr"ager eines \textbf{Talismans} weiß zwar nicht im Voraus, wie
lange die Wirkung des Talismans anhalten wird, aber sobald der
Talisman wirkungslos geworden ist, spürt er dies intuitiv

\textbf{Zaubersalze} k"onnen in mehreren Dosen ausgestreut werden. Der
Thaumaturg kann sowohl mehrere Dosen gleichzeitig aktivieren (kostet
je Dosis 1 AP, es wird nur ein EW:Zaubern durchgef"uhrt) als auch
kontrolliert eine Dosis nach der anderen aktivieren (eine Dosis pro
Runde). Eine Dosis Zaubersalz wiegt etwa ein halbes Gramm.

Bei der \textbf{Herstellung von Zaubermitteln} k"onnen auch mehrere
Anwendungen in einem Arbeitsgang hergestellt werden, wenn gen"ugend
Ausgangsmaterial vorliegt. Allerdings sollte die Anzahl
herzustellender Anwendungen innerhalb vern"unftiger Grenzen bleiben,
wie sie sich aus technischen oder rituellen Gr"unden ergeben k"onnten.
(Um den \emph{sense of wonder} nicht aus den Augen zu verlieren und
die Herstellung von Zaubermitteln zum rein (spiel-)technischen Vorgang
herabsinken zu lassen, ist es ohnehin sinnvoll, die
Herstellungsprozedur genauer auszuarbeiten. Dabei ergeben sich die
genannten Grenzen ganz von selbst. Im Extremfall ist ein Zaubermittel
einmalig, weil es schon bei der Zubereitung f"ur eine bestimmte
Situation oder Person angepa"st werden mu"s. Dies w"are z.\,B. bei einem
Liebestrank sehr gut denkbar. Die fantasievolle Ausarbeitung tr"agt
dann ihren Teil zum Rollenspiel bei.)

\textbf{Schutzrunen} [DFR\,215] "`bemerken"' den wahren Willen ihres
Besitzers ganz genau. Die in den FAQ geschilderten "`legitimen
Tricks"' sind nicht erlaubt. Wenn dem Besitzer in einem bestimmten
Moment lieber ist, daß das Opfer den geschützten Gegenstand berührt,
als daß es ihn nicht berührt, dann zündet die Schutzrune auch nicht.

In ein \textbf{Gro"ses Siegel} kann der Zauberer beliebig viele seiner
AP investieren. Dies ist dann von Bedeutung, wenn die AP-Kosten eines
Zaubers von Anzahl und Grad der Opfer abh"angen.

Wenn eine \textbf{Feuerperle} [DFR\,224] erfolgreich so geworfen wird,
daß die Explosion als \emph{Feuerkugel} ausgelöst wird, muß ein
EW:\emph{Zaubern} mit (in der Regel) Erfolgswert +20 durchgeführt
werden. Kritische Fehler und Erfolge werden wie bei Schutzrunen
behandelt.

\subsection{Zauberlieder}

Es gibt Zauberlieder, die erst wirken, wenn sie eine gewisse Zeit
gespielt werden. Bis zu diesem Zeitpunkt sind die potentiellen Opfer
zwar schon in den Bann der Musik gezogen, sie k"onnen aber durchaus
gegen den Barden vorgehen, wenn die Wirkungsart des Liedes das nicht
ausschließt und sie von der Gefahr wissen, die vom Spiel des Barden
ausgeht. Sinngemäß gilt entsprechendes für verwandte Magieformen
(Tänze etc.).

\subsection{DaiYao}

Bei den \textbf{DaiYao}-Zaubern muß der \textbf{XuéDsche} das zu
verzaubernde Wesen während der Zd immer wieder berühren (und nicht nur
kurz am Ende der Zd, wie das bei Rw Berührung sonst üblich ist).

Die DaiYao-Version von \emph{Bannen von Gift} stabilisiert das Opfer
10\,sec nach Beginn des Zaubervorgangs so weit, daß das Gift während
der Zd keinen weiteren Schaden erleidet. Mißlingt der Zauber, tritt
der verzögerte Schaden komplett unmittelbar nach Ende des
Zaubervorgangs ein. (Ohne diese Regelung wäre der Zauber gegen die
meisten Gifte sinnlos.)

\subsection{Strahlzauber}
\label{strahlzauber}

Sprücke mit Wirkungbereich \emph{Strahl} sorgen für einige
regeltechnische Verwirrungen. Die folgenden Anmerkungen versuchen,
diese aufzuklären.

Zunächst einmal ist zu beachten, daß es zwei Gruppen von
Strahlzaubersprüchen gibt, die sich recht grundsätzlich unterscheiden:
Die Sprüche der Gruppe A treffen automatisch (\emph{Blitze
  schleudern}, \emph{Donnerkeil}, \emph{Feuerlanze},
\emph{Frostball}), während bei den Sprüchen der Gruppe B zum Treffen
ein EW:Angriff benötigt wird (\emph{Dämonenfeuer}, \emph{Elfenfeuer},
\emph{Göttlicher Blitz}, \emph{Wasserstrahl}).

Die Regel, daß bei Verfehlung des Zieles andere Personen (gemeint sind
wohl generell Wesen oder auch Dinge) in der Schußbahn getroffen werden
können, findet nur für Gruppe B Anwendung. Bei Sprüchen der Gruppe A
kann das Ziel schließlich nicht verfehlt werden. Wenn der EW:Zaubern
mißlingt, findet der magische Angriff gar nicht erst statt. Gelingt
hingegen der WW:Resistenz, wird das Ziel ja dennoch getroffen, wenn
auch mit sehr viel geringerem Schaden. Dies entspricht der Regelung
bei Fernkampfangriffen, wo nach einer gelungenen Abwehr das Geschoß
ebenfalls nicht mehr andere schädigen kann (obwohl ja durchaus denkbar
ist, daß die erfolgreiche Abwehr für ein Ausweichen steht, das das
Geschoß mitnichten abfängt).

Für Sprüche der Gruppe A muß der Zauberer sich ein Wesen oder einen
Gegenstand als Ziel aussuchen, das oder den er dann wie üblich während
der gesamten Zauberdauer sehen muß (mit der ebenfalls üblichen
pragmatischen Tolerierung von sehr kurzen Unterbrechungen des
Blickkontakts durch Zwinkern oder wenn ein anderes Wesen zwischen
Zauberer und Opfer durch das Blickfeld läuft). Beispielsweise kann der
Zauberer einen 10-sec-Spruch nicht auf einen Schützen anwenden, der
sich nur kurz aus seiner vollständigen Deckung hinausbewegt, um einen
spontanen Schuß (nach [DFR\,235]) abzugeben. Bei einem regulären
Ferkampfangriff ist der Schütze aber lange genug sichtbar, um
verzaubert zu werden. Wenn die magische Energie schließlich
freigesetzt wird (nach einem gelungenen EW:Zaubern), trifft sie
unabwendlich ins Ziel. Selbst im Handgemenge kann niemand anders
getroffen werden. Das magische Geschoß bewegt sich so schnell, daß nur
unter außergewöhnlichen Umständen ein Gegenstand oder Wesen in die
Schußbahn gelangen kann, um das Geschoß abzufangen. (Befindet sich
schon während der Zauberdauer etwas oder jemand in der Schußbahn, kann
der Spruch gar nicht erst vollendet werden, weil der Zauberer das
Opfer nicht sehen kann.) Im übrigen \emph{muß} es auch immer ein
definiertes Ziel geben. Es ist nicht möglich, ungezielt "`ins Blaue"'
zu zaubern in der Hoffnung, durch Zufall irgendetwas zu treffen
(z.\,B. ungezielt \emph{Blitze schleudern} in eine Menschenmenge
oder in einen völlig dunklen Bereich hinein, aus dem der
Zauberer Geräusche gehört hat).

Ganz anders steht bei Gruppe B das Opfer während der Zauberdauer noch
gar nicht fest (mit \emph{Göttlichem Blitz} als Ausnahme, siehe weiter
unten). Das Ritual wird gewissermaßen vorab durchgeführt, was
prinizipiell wie ein Zauber mit Reichweite 0m auch in völliger
Dunkelheit passieren kann. Die eigentlichen Angriffe werden dann (mit
den in der Spruchbeschreibung aufgeführten Ausnahmen) nach den
üblichen Regeln für nicht-magische Angriffe durchgeführt (was
z.\,B. Angriffe in Dunkelheit angeht). \emph{Göttlicher Blitz} stellt
hier eine Ausnahme dar, da während der gesamten Zauberdauer das
zukünftige Opfer wie üblich sichtbar sein muß, dann aber später mit
normalen Angriffsregeln bekämpft wird.

Beim \emph{Funkenregen} gilt, daß ein Zauberer, der an die Illusion
glaubt, das Opfer des \emph{Funkenregens} nicht mit Sprüchen der
Gruppe A anvisieren kann und auch nicht während der Zauberdauer als
Opfer des \emph{Göttlichen Blitzes} auswählen kann. Andere Sprüche der
Gruppe B werden wie normale Angriffe behandelt und können
dementsprechend eingesetzt werden. \emph{Göttliche Blitze} können mit
den Abzügen für Fernkampf auf das Opfer des \emph{Funkenregens}
abgeschossen werden, wenn während der Zauberdauer des \emph{Göttlichen
  Blitzes} der \emph{Funkenregen} noch nicht aktiv war.

\subsection{Einzelne Spr"uche}

\textbf{Bannen von Gift} enthält Angaben über die Wirkdauer von
Giften, die in direktem Widerspruch zu [DFR\,107f] stehen. Letzteres
gilt.

Die \textbf{Goldene Bannsphäre} wirkt nur gegen Zombies, die wirklich
mit schwarzer Magie erschaffen wurden, also nicht gegen die
„üblichen“ mit \emph{Macht über den Tod} erschaffenen Zombies.

F"ur das Zaubersiegel \textbf{Beeinflussen} gelten die gleichen
Sonderregeln wie f"ur das Siegel \emph{Anziehen}.

Beim \textbf{Blitze schleudern} wird der WW:Resistenz für jeden Blitz separat
gewürfelt, auch wenn mehrere Blitze sich gegen dasselbe Opfer richten.

Wird der nötige Mindestschaden von 1\,LP für \emph{Meucheln}
[DFR\,160ff] oder gezielte Angriffe (siehe Abschnitt \ref{gezielt})
allein duch \textbf{Blutmeisterschaft} verhindert, verliert das Opfer
zwar in der Tat keine LP, ansonsten wird aber so verfahren, als ob
1\,LP Schaden eingetreten wäre.

Beim \textbf{Dämonenfeuer} wird der Angriffsbonus sowohl für den
Einsatz als Lichtschwert wie beim Einsatz als Thaumagralzauber
angerechnet, nicht aber beim Einsatz als Lichtpfeile. Der
Schadensbonus wird nur beim Einsatz als Thaumagralzauber
angerechnet. Beim Thaumagralzauber sind keine Rundumschläge wie mit
dem Lichtschwert möglich. Beim Rundumschlag mit dem Lichtschwert
werden keine WM--4 angerechnet, und der Schlag findet zum normalen
Zeitpunkt gemäß Angriffsrang statt. (Der Waffenrang des Lichtschwertes
ist 0.) Generell wird das Lichtschwert auch nicht zweihändig geführt,
so daß keine WM-2 auf den WW:Abwehr angerechnet werden.

Das Zaubersiegel \textbf{Dinge wiederfinden} wird ausnahmsweise nicht
durch Sichtkontakt aktiviert, sondern beim Anpeilen. Jedes Siegel kann 
nur einmal angepeilt werden und verschwindet dann. Es ist m"oglich,
mehrere Siegel auf einem Gegenstand anzubringen, die getrennt
voneinander angepeilt werden k"onnen. Zum Aufbringen des Siegels auf
einen magischen Gegenstand und für die Wirkung des Siegels ist es
unerheblich, ob der Gegenstand schon längere Zeit vom Thaumaturgen
besessen wurde. (Dies ist im Prinzip genau so in den FAQ geregelt,
allerdings sehr verquer formuliert.)

Für \textbf{Elfenfeuer} gelten sinngemäß die gleichen Anmerkungen
wie für \emph{Dämonenfeuer} (s.\,o.).

Wird \textbf{Entgiften} auf Fl"ussigkeiten angewandt, die in großer
Menge einheitlich vergiftet vorliegen, kann pro Anwendung maximal
1\,hl Fl"ussigkeit entgiftet werden.

Das \textbf{Erkennen der Aura} spricht auf Zauberwerk an, wenn es eine ABW von
\emph{höchstens} 10 hat (nicht \emph{mindestens}).

Das Zaubersiegel \textbf{Erkennen von Leben} kann der Thaumaturg auf
sich und andere anwenden, indem er das Siegel auf die Schl"afe des zu
Verzaubernden auftr"agt.

\textbf{Flammenklinge} ist auf Waffen anwendbar, die mit den
Fertigkeiten \emph{Dolch}, \emph{Kurzschwert} oder \emph{Langschwert}
geführt werden, nicht aber auf andere Einhandschwerter oder auf
Zweihandschwerter.

Der \textbf{Funkenregen} verpufft wirkungslos, wenn sein Primäropfer
nicht an die Illusion glaubt (weil der WW:Resistenz geglückt ist oder
der Zauberer sich selbst verzaubert hat). Siehe auch Anmerkungen im
Abschnitt \ref{strahlzauber}.

Beim \textbf{Heilen von Krankheiten} kann es bei mißglückter Heilung
ansteckender Krankheiten genauso wie beim \emph{Erkennen von
  Krankheiten} zu einer Ansteckung des Zauberers kommen. Der Zauber
eignet sich nicht zur Heilung von Geisteskrankheiten (dafür gibt es
\emph{Seelenheilung}). Besonders schwer zu heilende Krankheiten können
sich in einem Zauberduell wehren (wie bei \emph{Seelenheilung}). Der
Zaubern-Wert der Krankheit hängt von ihrer Schwere ab. Ein verlorenes
Zauberduell darf wie üblich wiederholt werden.

\textbf{Schlaf} wirkt nicht gegen mögliche Opfer, die sich im Kampf
oder in einer anderen lebensbedrohlichen Situation befinden.
% Backport von M5.
Die Magie des \emph{Schlafdorns} ist allerdings so stark, daß hierbei
das Opfer in jeder Situation einschläft.
% Der Schlafdorn bei M5 wirkt Zauberschlaf, nicht "normalen" Schlaf.

Das Zaubersiegel \textbf{St"arke} wird auf einen Arm oder ein Bein des 
zu Verzaubernden aufgetragen (wirkt aber dennoch auf den ganzen
K"orper). Auch bei Stärkesiegeln kann immer nur ein Siegel pro Runde ausgelöst
werden. Es kann also nur ein Wesen pro Runde verzaubert werden.

Die durch einen Runenstab \textbf{Stille} erzeugte Zone der Stille
bewegt sich nicht mit dem Zauberer, sondern mit dem Runenstab.

% Folgendes Gestrichen, weil es dafür jetzt Runenbolzen gibt:
%
% Der \textbf{Todeshauch} eines Runenstabes breitet sich 1min lang vom
% Stab aus aus. Der Stab kann also w"ahrend der Wirkungsdauer
% weggeworfen oder -geschossen werden, um an seinem Zielort den
% Giftnebel auszubreiten.

Die mit dem Zaubersiegel \textbf{Unsichtbarkeit} verzauberten Personen 
m"ussen sich auf den Zauber konzentrieren, um unsichtbar zu bleiben.

Die \textbf{Vision} benötigt auch als Wundertat die angegebene
Materialkomponente.

Der Runenstab \textbf{Wasserstrahl} wirkt nur solange, wie der
Anwender den Runenstab in der Hand behält und sich auf den Strahl
konzentriert.

Die \textbf{Wundrune} ist von der Regelung zur Behandlung tödlich
Verwundeter ausgenommen [DFR\,102]. Gewissermaßen wird der größte Teil
der Zauberdauer bei der Wundrune durch die Vorbereitung eingenommen,
bei der der Verzauberte zwar zugegen sein muß, aber sich noch nicht
"`in Behandlung"' befindet. Erst fast am Ende der 30min Zauberdauer
wird das hergestellte Pulver über die Verletzung gestreut und
entfaltet seine Wirkung. Ist die zu verzaubernde Figur zu diesem
Zeitpunkt bereits tot, hilft der Zauber nicht mehr. Zum Ausgleich kann
eine Wundrune (im Gegensatz zu den normalen Heilzaubern) beliebig
häufig hintereinander eingesetzt werden. (Siehe auch oben die
Einschränkungen bei der Heilung ernster Verletzungen.)

Das Zaubersiegel \textbf{Zauberschlo"s} verschwindet, sobald es zu
wirken anf"angt.

\section{Sonstiges}

\subsection{Geld, Gold, Edelsteine\dots}

\emph{Anmerkung: Die Geldmengen bei \textsc{Midgard} wirken oft
  astronomisch, die Kaufkraft des Goldes l"acherlich. Als Ausweg wird
  h"aufig eine "`W"ahrungsreform"' gew"ahlt. Hier wird ein anderer Weg
  gegangen, der nur wenige "Anderungen ben"otigt, aber einiger
  Erl"auterungen bedarf.}

Die (Standard-)Geldst"ucke sind etwa so gro"s wie ein Centst"uck. KS
bestehen haupts"achlich aus Kupfer, SS aus Silber, GS aus 60\,\%\ Gold
und 40\,\%\ Silber. Ein GS wiegt dann 3,2g, ein SS 2,1, ein KS 1,8g.
(Also nicht wie bei [DFR\,323] angegeben, wo ein GS so gro"s wie ein
50-Cent-St"uck ist und 10g wiegt, was übrigens auch nicht konsistent
ist, denn so eine Münze wöge 14,3g.) Damit sind die
\textbf{Gewichtsprobleme} stark reduziert. Edelsteine, Schmuck und
sonstige Wertsachen k"onnen au"serdem als Zahlungsmittel mit h"oherer
Wertdichte dienen. Allerdings bedarf es realistisch betrachtet hoher
Qualifikation, den Wert dieser Wertgegenst"ande zuverl"assig zu
sch"atzen (oder erfolgreiches Anwenden von \emph{Schätzen}). In der
Regel werden SpF durch ihre Unwissenheit recht hohe Wertverluste
sowohl beim Kauf wie beim Verkauf erleiden.

\fbox{\parbox{0.98\textwidth}{%
  \textbf{Beispiel:} \emph{Die Gruppe hat von ihrem letzten
    Abenteuer eine ganze Truhe mit GS mitgebracht. Bei einem
    Edelsteinh"andler kaufen sie f"ur die rund 10.000\,GS einen
    wundersch"onen gro"sen Edelstein, der in Wirklichkeit aber nur
    6.000\,GS wert ist. Ein paar Monate und viele
    Kilometer sp"ater wollen sie den Edelstein wieder zu Geld machen 
    und bieten ihn einem H"andler f"ur 11.000\,GS an. Der H"andler
    bemerkt sofort, da"s die Gruppe keine Ahnung hat, und erkl"art
    ihnen freundschaftlich, da"s sie sich "ubers Ohr hauen
    lassen haben und da"s ihr Stein keine 3.000\,GS wert ist, er
    aber als echter Freund bereit ist, ihnen 3.500\,GS zu
    zahlen. -- N"achstesmal sind die SpF kl"uger und ziehen einen
    unabh"angigen Experten zurate. Sie werden nicht mehr "ubers Ohr
    gehauen, aber der Experte verlangt f"ur seine Dienste eine
    Pr"amie von 5\,\%\ des Sch"atzwertes.}
}}

Es empfiehlt sich, derartige Situationen nur dann in dieser Weise
auszuspielen, wenn es die Spielsituation verlangt. Ansonsten sollten
pauschal Abz"uge vorgenommen werden. Im Endeffekt mu"s nur klarwerden,
da"s Transport und Tausch von Geld und Wertsachen nicht ohne Risiko
und Verlust m"oglich ist. Auch W"ahrungsprobleme (ein albischer Oring
ist im Ausland lange nicht so viel Wert wie ein valianischer Orobor;
eine W"ahrung, deren Wert nicht ann"ahernd durch den Materialwert
gedeckt ist, ist in L"andern, wo diese W"ahrung nicht bekannt ist, gar
nichts wert, usw.) sollten bewu"st sein, ohne sie in all ihren
nervenden Details auszuspielen. F"ur nicht explizit ausgespielte
Lernphasen sind die beschriebenen Verluste einer der Gr"unde daf"ur,
warum die Lernkosten so hoch erscheinen (s.\,o.). In Lernphasen sind
also keine weiteren Abz"uge aufgrund von Tauschverlusten mehr
erforderlich. (Es wird allerdings der Einfachheit darauf verzichtet,
SpF, die \emph{Schätzen} beherrschen, einen pauschalen Nachlaß auf die
Lernkosten zu gewähren. Der pauschale Nachlaß von 10\,\%, den
\emph{Geschäftstüchtigkeit} bewirkt, erklärt sich durch die breitere
Anwendbarkeit dieser sehr mächtigen Fertigkeit.)

Der Materialwert der M"unzen ist etwa halb so hoch wie ihr Nennwert.
So kann man den Materialwert der Metalle absch"atzen. 1\,kg Silber ist
ca. 24\,GS wert (irdischer Wert: In den 1990ern um 150\,\euro, 2009 um
400\,\euro).  1\,kg der M"unzlegierung, aus der GS bestehen (60\,\%\
Gold, 40\,\%\ Silber), sind ca. 160\,GS wert. Das Silber davon hat
einen Wert von ca. 10\,GS, bleiben f"ur das Gold 150\,GS. Allerdings
wird das Gold durch Aufreinigung wertvoller. "`Normales"'
\textbf{Handelsgold} hat einen Goldanteil von ca. 90\,\% (was dem auf
der Erde häufig verwendeten 22-karätigem Münzgold entspricht). 1\,kg
dieses Goldes aus einer in irgendeiner Form zertifizierten Quelle hat
einen Wert von gut \textbf{250\,GS}. Hochaufgereinigtes und als
solches zertifiziertes Feingold (\textbf{Alchimistengold}) hat einen
Wert von mindestens \textbf{500\,GS je kg}. (Zum Vergleich: Der
irdische Goldpreis unterliegt starken Schwankungen, bedingt durch
Spekulation, wechselnde Bestrebungen, Währungen durch Gold zu decken,
und vieles mehr.  In den 1950er und 1960er Jahren lag der Goldpreis
recht stabil bei knapp 3.000\,\euro/kg. Nachdem die Golddeckung des
US-\$ aufgegeben wurde, schwankte der Preis extrem stark mit dem
vorläufigen H"ohepunkt Anfang 1980 bei fast 25.000\,\euro/kg. Nach
einer Talsohle bei etwa 8.000\,\euro/kg in den 1990ern erreichte der
Goldkurs Ende 2009 fast wieder seinen Rekordkurs (wobei all diese
Zahlen nicht inflationsbereinigt sind).) Stellt man sich vor, da"s ein
GS etwa einem Gegenwert von 10\,\euro\ entspricht, so erh"alt man
(bezogen auf die irdischen 1990er Jahre) einen Goldpreis, der leicht
unter dem irdischen liegt, und einen Silberpreis, der leicht "uber dem
irdischen liegt, eine Relation, die im Hinblick auf die Aussage, da"s
Gold auf Midgard h"aufiger ist als auf der Erde, durchaus
konsistent erscheint. Allerdings ist mit derartigen Vergleichen
"au"serst vorsichtig umzugehen. Die Rolle des Geldes in einer
altert"umlichen Fantasywelt ist nur sehr bedingt vergleichbar mit der,
die es in unserer heutigen Welt spielt. In einer bestimmten Situation
kann die Kaufkraft eines GS der eines einzigen Euro (oder noch
weniger, schlie"slich kann man weder Geld noch Gold essen) entsprechen
oder auch der von 100\,\euro.



\subsection{Sex, Liebe, Leidenschaft\dots}
Mit der 4.~Auflage des Regelwerks sind die Stellen, die den Eindruck erwecken,
als ob \textsc{Midgard} nur Heterosexualit"at kenne, deutlich weniger geworden,
aber leider noch nicht ganz verschwunden (z.\,B. [DFR\,32], [ARK\,179],
[BES\,323]). So ist jetzt etwas klarer, daß es natürlich auch Homo- und
Bisexualit"at (und wohl auch Asexualit"at) gibt. Wenn derartige
zwischenmenschliche Angelegenheiten regeltechnisch behandelt werden m"ussen,
so geht der SpL flexibel vor und vergibt passende WM auf entsprechende
W"urfe. Es ist immer m"oglich, da"s ein Charakter gegen seine bevorzugte
sexuelle Orientierung handelt, wenn die Umst"ande entsprechend extrem sind.

\fbox{\parbox{0.98\textwidth}{%
    \textbf{Beispiel:} \emph{Die schwarze Hexe Xenia macht sich an
      Ipak, den gutaussehenden Gl"ucksritter der Abenteurergruppe,
      heran. Sie hat vor, die Gruppe zu schw"achen, indem sie Ipak
      mittels eines Liebeszaubers an sich binden und in einem
      g"unstigen Moment beseitigen wird. Was sie nicht wei"s: Ipak ist
      schwul.  Verf"uhrungsversuche und auch Liebeszauber des
      weiblichen Geschlechts werden nicht v"ollig aussichtslos sein
      (ein Gl"ucksritter bleibt ein Gl"ucksritter und ist immer offen
      f"ur neuartige Erfahrungen, Xenia ist obendrein extrem
      gutaussehend und charismatisch), aber doch um ein geh"origes
      Ma"s erschwert. Der SpL entscheidet also, Ipak WM+8 auf seinen
      WW:Resistenz gegen Geistesmagie zu gew"ahren. -- Nachdem Xenia
      von Ipak einen freundschaftlichen Korb erhalten hat, probiert
      sie es mit dem zweiten m"annlichen Mitglied der Gruppe, dem
      Priester Fares. Fares lebt gem"a"s den Vorschriften seiner
      Glaubensgemeinschaft streng keusch. Da er wahrhaft
      gottesf"urchtig ist, ist dies nicht nur ein Lippenbekenntnis,
      sondern seine tiefste "Uberzeugung. Xenia wird es mit ihm
      "ahnlich schwer haben wie mit Ipak, selbst wenn Fares eigentlich
      heterosexuell ist (was aber aufgrund des "uberzeugten Z"olibats
      relativ unwichtig ist). Eine tiefe "Uberzeugung wiegt genauso
      schwer wie eine biologische oder psychische Veranlagung.}  }}

\appendix

\section*{Anhang}

\section{Handlungsunf"ahigkeit und Wehrlosigkeit}
\label{behindert}

Abwehr ist regeltechnisch keine Handlung. Handlungsunf"ahigkeit und
Wehrlosigkeit sind daher unterschiedliche Konzepte, auch wenn in der
Praxis beide häufig gleichzeitig auftreten.

Eingeschränkte Handlungsfähigkeit führt zu negativer WM auf die
eigenen Handlungen. Bei Handlungsunfähigkeit kann die Figur gar keine
Handlungen mehr durchführen.

Eingeschränkte Fähigkeit zur Abwehr führt zu negativer WM auf die
eigene Abwehr oder zu positiver WM auf den Angriff des Gegners (wobei
bei gleichem Betrag der WM die negative WM auf die Abwehr weniger
Auswirkungen hat als die positive WM auf den Angriff). Bei
Wehrlosigkeit kann die Figur gar keine Abwehr mehr
durchführen. Zusätzlich erhält der Gegner WM+4 auf seine Angriffe oder
trifft in extremen Fällen sogar automatisch kritisch.

Tabelle \ref{tab:handlung} zeigt beispielhaft, wie Spielsituationen
auf diesen beiden Achsen eingeordnet sind.

\pagebreak

\begin{table}[htbp]
{\footnotesize
  \begin{center} 
    \renewcommand{\arraystretch}{1.2}
    \begin{tabu}{|p{2.4cm}||p{4.3cm}|p{4.1cm}|p{4.5cm}|}
      \hline
      Einschränkung der & \multicolumn{3}{c|}{Einschränkung der Handlungsfähigkeit}\\
      Abwehrbereitschaft & keine & WM--4 auf Angriff & keine Handlung \\
      \hline
      \hline
      keine &
      \begin{itemize}
      \item Standardsituation
      \item leichte Überraschung (Opfer verliert Initiative)
      \end{itemize} &
      \begin{itemize}
      \item spontaner Angriff
      \item Rundumschlag
      \item Opfer hat mindestens die Hälfte der LP verloren (WM--2)
      \end{itemize} & 
      \begin{itemize}
      \item w"ahrend 30\,min (bzw. 60\,min) nach kritischem Kopftreffer 74--80
      \item Schock nach kritischem Treffer 11--20, Zerstörung des
        Thaumagrals oder Peitschenhieb
      \item spezielle Situation im Faustkampf [DFR\,197f]
      \item für eine Runde nach krit. Fehler Abwehr 51--60
      \end{itemize} \\
      \hline
      WM--4 auf Abwehr &
      \begin{itemize}
      \item Angriff mit zweihändiger Hiebwaffe oder schlagender
        Stangenwaffe (WM--2)
      \end{itemize}      & 
      \begin{itemize}
      \item überstürzter Angriff (Ang. WM--6, Abw. WM--2)
      \item Kampf gegen Unsichtbare (Ang. WM--2)
      \end{itemize}      &
      \begin{itemize}
      \item schwere "Uberraschung
      \item die erste Runde, in der Opfer während der Aufwachzeit
        [DFR\,103] angegriffen wird
      \end{itemize} \\
      \hline
      WM+4 auf Nahkampfangriff des Gegners &
      \begin{itemize}
      \item Opfer im Handgemenge (WM+4 auch bei Fernkampf)
      \item Opfer flieht panisch
      \item Opfer wird von hinten oder aus erhöhter Position
        angegriffen (WM+2)
      \item Opfer im Zangenangriff (WM+1)
      \end{itemize} &
      \begin{itemize}
      \item Opfer am Boden liegend
      \end{itemize} &
      \begin{itemize}
      \item Opfer wird im Handgemenge festgehalten
      \end{itemize}
      \\
      \hline
      WM+4 auf Nahkampfangriff des Gegners, keine Abwehr &
      \begin{itemize}
      \item "`klassische"' Wehrlosigkeit [DFR\,95]
      \item schwere Beinverletzung (B6)
      \item Ahnungslosigkeit (\emph{Meucheln} möglich)
      \item Opfer wird in Schach gehalten [DFR\,95]
      \item für Zusatzangriff nach krit. Fehler Abwehr 61--70
      \end{itemize}&
      \begin{itemize}
      \item Opfer hat 0\,AP [DFR\,101]
      \end{itemize}& 
      \begin{itemize}
      \item Opfer hat 1-3\,LP (B4)
      \item für eine Runde nach krit. Fehler Abwehr 81--90
      \end{itemize}\\
      \hline
      automatischer kritischer Treffer\footnotemark&
      ---&
      ---&
      \begin{itemize}
      \item Bewegungslosigkeit des Opfers
      \item Opfer hat 0\,LP (B0)
      \item Opfer bewußtlos
      \item Opfer schläft (vgl. Anmerkungen zu \emph{Schleichen} oben)
      \end{itemize} \\
      \hline
    \end{tabu}
    \renewcommand{\arraystretch}{1.5}
    \caption{Handlungsunf"ahigkeit und Wehrlosigkeit}
    \label{tab:handlung}
  \end{center}
}
\end{table}

\footnotetext{Bei EW:Meucheln (mit
  WM--Grad, wenn Reflexe noch eine Rolle spielen können) wird
  Opfer get"otet. Ist der Angreifer ungestört und hat genug Zeit
  (in der Regel au"serhalb von Aktionsphasen), tötet er das Opfer
  automatisch.}

\end{document}


% \fbox{\parbox{0.98\textwidth}{%
%   \textbf{Beispiel:} \emph{
% }}}
